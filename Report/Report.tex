% PRÄAMBEL
% ***************************************************************************************************

\documentclass[smallheadings,headsepline,11pt,oneside,a4paper]{scrbook}
%\documentclass{article}

% Hier gibt man an, welche Art von Dokument man schreiben möchte.
% Möglichkeiten in {}: scrartcl, scrreprt, scrbook, aber auch: article, report, book

\usepackage[american]{babel} 		% ermöglicht deutsche Silbentrennung und direkte Eingabe von Umlauten, ...
\usepackage[utf8]{inputenc} 	% teilt LaTeX die Texcodierung mit. Bei Windowssystemen: ansinew
\usepackage[T1]{fontenc} 		% ermöglicht die Silbentrennung von Wörtern mit Umlauten

\usepackage[usenames,dvipsnames]{color}

% PDF wird mit Lesezeichen (verlinktes Inhaltsverzeichnis) versehen (bei Betrachtung mit Acrobat Reader sichtbar)
\definecolor{farbelink}{rgb}{0,0,0}		%Definition der Linkfarbe für das PDF-File
\usepackage[pdftex,colorlinks=true,urlcolor=farbelink,linkcolor=farbelink,citecolor=farbelink]{hyperref}

\usepackage{lscape}

\usepackage{amssymb,amsmath}

\usepackage{geometry} 
\geometry{a4paper} 
\usepackage[parfill]{parskip}    								% Activate to begin paragraphs with an empty line
\usepackage[pdftex]{graphicx}
\usepackage{epstopdf}
\DeclareGraphicsRule{.tif}{png}{.png}{`convert #1 `dirname #1`/`basename #1 .tif`.png}
\usepackage[margin=10pt,font=small,labelfont=bf]{caption}		%Package für kleinere Schrift unter den Bildern

\usepackage[usenames]{color}

\usepackage{listings} 									%Einfügen von Programmcode
\lstset{numbers=left, numberstyle=\tiny, numbersep=5pt} 			%mit Zeilennummerierung links
\lstset{language=[77]Fortran}								%Programmiersprache: Fortran77


%text with 2.5cm on both sides (A4=210mm)
\setlength{\textwidth}{160mm}
\setlength{\evensidemargin}{0in}
\setlength{\oddsidemargin}{0in}
\setlength{\columnsep}{0.25in}
%
%text with 2.5cm on top and 2.5cm on bottom (A4=297mm)
\setlength{\textheight}{247mm}
\setlength{\topmargin}{0pt}
\setlength{\headsep}{25pt}
\setlength{\headheight}{0pt}


%\typearea{12} % Breite des bedruckten Bereiches vergrössern (funktioniert nur in \documentclass mit: scrreprt, scrartcl, scrbook)

\clubpenalty = 10000 		% schliesst Schusterjungen aus
\widowpenalty = 10000 		% schliesst Hurenkinder aus


\usepackage{fancyhdr}
\pagestyle{fancy}			%Selbst gestaltete Kopf-/Fusszeilen

%%% Fancy Header %%%%%%%%%%%%%%%%%%%%%%%%%%%%%%%%%%%%%%%%%%%%%%%%%%%%%%%%%%%%%%%%%%
% Fancy Header Style Options
\fancyhf{}
\fancyfoot{} 

\renewcommand{\chaptermark}[1]{         		% Lower Case Chapter marker style
\markboth{\chaptername\ \thechapter.\ #1}{}} 	%
\renewcommand{\sectionmark}[1]{         		% Lower case Section marker style
\markright{\thesection.\ #1}}         			%
\fancyfoot[R]{\thepage}    					% Page number in the right on every page
\fancyhead[L]{\leftmark}      				% Chapter in the right on even pages
\fancyhead[L]{\rightmark}     				% Section in the left on odd pages
\renewcommand{\headrulewidth}{0.3pt}   		% Width of head rule

\fancypagestyle{plain}{%
\fancyhf{} 								% clear all header and footer fields
\fancyfoot[R]{\thepage}    					% Page number in the right on every page
\fancyhead[L]{\leftmark}       				% Chapter in the right on even pages
\renewcommand{\headrulewidth}{0.3pt}}   	% Width of head rule


\setlength{\textheight}{21cm}				%Vergrössert die Höhe des bedruckbaren Bereichs

%%%%%%%%% PERSONAL PACKAGES %%%%%%%%%%%%%%%%%%
\usepackage{natbib}
%\usepackage{cite}
\usepackage{physics}
\newcommand{\equref}[1]{(\ref{#1})}
%\usepackage[ruled, vlined ]{algorithm2e}
\usepackage{algpseudocode}
\usepackage{algorithm}
\usepackage{caption}
\usepackage{subcaption}
%%%%%%%%% PERSONAL PACKAGES %%%%%%%%%%%%%%%%%%
\begin{document}

\frontmatter							%römische Seitenzahlen

% TITELSEITE
% ***************************************************************************************************
% use FS or HS to identify the semester, FS (Fruehling) for spring semester, HS (Herbst) for fall semester

\begin{titlepage}
\begin{center}
    \vspace*{1cm}
    {\huge \bfseries  Impact of Molecular Diffusion on Transport in Natural Porous Media\\ }
    \vspace{2cm}
    {\large 
	Antoine Hoffmann\\
	~\\
	Master in Computational Science and Engineering, EPFL\\
	\vspace{3.5cm}
	Summer Semester 2019\\
	~\\
	Institute of Fluid Dynamics\\
	ETH Zürich\\
    }

\vspace{\stretch{1}}



{\large
	Supervisor: Daniel Meyer-Massetti\\[\baselineskip]
	ETHZ Professor: Patrick Jenny\\[\baselineskip]
	EPFL Professor: François Gallaire
}
\end{center}

\vspace*{2cm} % a bit of space at the bottom of the page

\end{titlepage}

%\clearpage\null


% Einfügen einer leeren Seite nach dem Titelblatt
\begin{titlepage}
\thispagestyle{empty}
\newpage
\mbox{}
\end{titlepage}


% ABSTRACT
% ***************************************************************************************************
% ABSTRACT
% ***************************************************************************************************
%mit horizonztalen Linien
\begin{tabular}{p{0.9\textwidth}}
\chapter*{Abstract}

This Master thesis in computational science and engineering depicts the issues rising from transport in porous media and study the effect of molecular diffusion in an previously developed Lagrangian numerical model. \\
A short literature review is presented with a comparison between three recent papers (\citet{Puyguiraud2019}, \citet{Dentz2017}, \citet{Meyer2016}).
Then, numerical simulations have been performed on the ETH Zürich EULER clusters in order to reproduce the results of \citet{Meyer2016} at purely advective regime ($\mathrm{Pe}=\infty$).
After this validation, these results have been extended to advection diffusion regimes ($\mathrm{Pe}<\infty$) where molecular diffusion appeared to significantly reduce the stagnation of the particle at $\Pe<10^4$.
An increase in the averaged Lagrangian velocity magnitude has also been observed for $10^2\leq\Pe\leq10^4$ demonstrating that a small amount of molecular diffusion frees particles from the boundary layers, introducing a physical bias in the velocity sampling.
At smaller Péclet values, the Lagrangian velocity histogram tends to converge to the same distribution as in the fully advective regime since a strong molecular diffusion is able to reroute the slow moving particles just as well as the fast ones.\\
Finally, a Péclet dependent probability to effectuate a random diffusive jump has been added to the stochastic model of \citet{Meyer2016}.
This hybrid model performed well at short time range but need surely more development to catch larger time dispersion behavior.
\end{tabular}



% INHALTSVERZEICHNIS
% ***************************************************************************************************

\tableofcontents
% Dieser Befehl erstellt das Inhaltsverzeichnis. Damit die Seitenzahlen korrekt sind, muss das Dokument zweimal gesetzt werden!

%\listoffigures


% HAUPTTEIL
% ***************************************************************************************************
% Folgende Befehle stehen für die Gliederung zur Verfügung: \chapter \section \subsection \subsubsection \paragraph
% Für Anführungszeichen: "` (CH-Tastatur: ZUERST: Shift-Tast und Taste 2, DANN: Shift-Taste und Taste ^)
% Für Schlusszeichen: "' (CH- Tastatur: ZUERST: Shift-Tast und Taste 2, DANN:  '-Taste, rechts neben der Null)
% Für Neuen Abschnitt: Eine Zeile leer lassen.


\mainmatter	%arabische Seitenzahlen


%1. Einführung
%------------------------------------------------------------------------------------------------------------------------------
\chapter{Introduction}
%%Contextual introduction
\begin{figure}
	\centering
	\includegraphics[scale=0.6]{Figures/bentheimer300_cuts.png}
	\caption{(left) Side cut of x-ray scanned sample of Bentheimer sandstone with $300^3$ voxel resolution and $L\approx0.3$mm. (right) z-averaged porosity for the same sample. Black regions show entirely solid column along z-axis.}
	\label{fig:bentheimer300cuts}
\end{figure}
Diffusion in porous media is a common but complex natural phenomenon occurring in a wide domain of application, from carbon storage to advanced oil extraction and medicine metabolization.
Despite of widely used macroscopic description of transport such as the Fick's law of diffusion (\citet{Fick1855}) and Darcy law (\citet{Darcy1856}), complex porous media can present behaviors that requires a microscopic approach in order to predict correctly the transport dynamic.
For example, the Bentheimer sandstone presents, at microscopic scale, very tortuous cuts with a small porosity (c.f. Figure \ref{fig:bentheimer300cuts} left).
The complexity of this sample is also put in evidence on Figure \ref{fig:bentheimer300cuts} (right) by plotting the percentage of void voxel along z-axis which demonstrates the existence of completely blocked area in the sample.
Flows among such media are driven by the pore-scale geometry and their description using simple macroscopic laws constitute an active research topic nowadays.\\
As an illustration, \citet{Dentz2017} demonstrated how the macroscopic advection-diffusion transport law
\begin{equation}
	\pdv{\bar{c}(x,t)}{t}+U\pdv{\bar{c}(x,t)}{x}-\mathcal{D}\pdv[2]{\bar{c}(x,t)}{x}=0
\label{eq:adv_diff}
\end{equation}
failed to describe the diffusion of the macroscale concentration $\bar{c}(x,t)$ for a mean flow $U$ and a macroscopic diffusion coefficient $\mathcal{D}$ by comparison to direct numerical simulation (DNS) data (\citet[Fig. 3]{Dentz2017}).\\


\section{General numerical approach of the problem}
In this article as in many other recent works, numerical approaches have been adopted in order to bridge the gap between microscopic behavior of the flow and macroscopic diffusion as \citet{Meyer2016}, \citet{Dentz2017} and \citet{Puyguiraud2019}. In their works, they all used roughly the following methodology.\\
\textit{Sample scanning}\\
First of all, a digital porous domain is obtained either synthetically (\cite{Dentz2017}) or by high resolution tomography of physical sandstone samples (\cite{Meyer2016} and \cite{Puyguiraud2019}).
Hence, a scalar field  $\rho_d(\textbf{x})$ on a domain $\Omega_d$ can be defined as 
\[
\rho_d(\textbf{x})=\left\{
\begin{array}{ll}
1 &  \textrm{if voxel at $\textbf{x}$ is solid} \\
0 & \textrm{otherwise}
\end{array} 
\right.
\]
In the wide variety of porous media, samples are usually differentiated thanks to the following features : their average pore length $L$, their porosity $\phi$ that expresses the ratio between solid and void volume, i.e.
\[
\phi = \frac{\int_{\Omega_d}\rho(\textbf{x})d\textbf{x}}{\int_{\Omega_d}d\textbf{x}}
\] 
and their tortuosity $\kappa$ that indicates how much paths in the media are twisted by comparing the total length of a path $l(t)$ with a linear travel distance $l_{\hat{\textbf{x}}_i}(t)$ in a arbitrary direction represented by a normal vector $\hat{\textbf{x}}_i$, i.e. 
\[
\kappa = \frac{l(t)}{l_{\hat{\textbf{x}}_i}(t)}
\] 
As in most cases, we will consider a cubic domain of size $L_0$, i.e. $\Omega_d=[0,L_0]^3$, and use the Cartesian spatial coordinates $\textbf{x}=(x_1,x_2,x_3)^T$, i.e. $\textbf{x}=x_1\hat{\textbf{x}}_1+x_2\hat{\textbf{x}}_2+x_3\hat{\textbf{x}}_2$ in terms of the corresponding normalized directional vectors.\\
It is worth noting that the domain must balance two contradicting requirements. On one hand, it has to be large enough to be a representative elementary volume (REV) of the sample, i.e. the frequency of every possible configuration of the porous network should be measurable.
This property can be verified by increasing the size of an REV considered sample and verifying that the average properties do not change.
However, on the other hand, the domain should remain as small as possible to spare computational resources in the upcoming numerical treatments.\\
\textit{Solving the stokes flow}\\
Once an REV of the porous sample has been digitalized, a numerical solver is used to solve the incompressible Stokes equation,
\begin{equation}\label{eq:stokes}
\begin{array}{c}
	\mu\nabla^2\textbf{u}(\textbf{x}) = \nabla p(\textbf{x}) \quad\textrm{and}\quad \nabla\cdot\textbf{u}(\textbf{x})=0\quad \forall \textbf{x}\in\Omega_d\\
	\textrm{s.t.}\quad p(L_0,x_2,x_3) - p(0,x2,x3) = \Delta\textrm{P} \quad \forall (x_2,x_3)\in[0,L_0]^2.
\end{array}
\end{equation}
where $\nabla = (\pdv{x_1},\pdv{x_2},\pdv{x_3})^T$ the Cartesian gradient, $\mu$ the fluid viscosity and $p(\textbf{x})$ the pressure field.
The pressure drop $\Delta\textrm{P}$ applied on two opposite faces of the cubic porous domain is meant to induce a mean flow velocity in the $\hat{\textbf{x}}_1$ direction, 
\[
U=-\frac{L \phi \Delta\textrm{P}}{\mu}.
\]
One often refers to the velocity $U$ with the help of the Péclet number that expresses the ratio between advection due to the flow and molecular diffusion, i.e.
\begin{equation}\label{eq:peclet}
\textrm{Pe}=\frac{U\,L}{D_m}
\end{equation}
with $D_m$ the molecular diffusion coefficient.
Characteristic times can be defined as the macroscopic advection time $\tau_a=U/L$ (often used as reference time $T\equiv\tau_a$) and the microscopic diffusive time $\tau_D=L_0^2/D$.\\
The numerical solution of (\ref{eq:stokes}) allows to obtain a precise map of the Eulerian velocity field $\textbf{u}$ in the digital porous domain. This enables to define more clearly the tortuosity of the studied domain as
\begin{equation}\label{eq:tortuosity}
\kappa=\frac{\int_0^t|\textbf{u}(t)|dt}{\int_0^t \textbf{u}(t)\cdot\hat{\textbf{x}}_idt}.
\end{equation}
By definition, the REV property ensures that the directional vector in \eqref{eq:tortuosity} can be arbitrary chosen without changing the value tortuosity.\\
\textit{DNS of particles in the flow}\\
Once an microscopic numerical solution of the flow in the medium is obtained, the displacement of $N_p$ particles injected at the inflow plane is simulated with another DNS using particle tracking methods as \citet{Pollock1988} to solve
\begin{equation}\label{eq:traject}
\dv{\textbf{x}_i(t)}{t}=\textbf{u}[\textbf{x}_i(t)]\quad\forall i\in\{1,2,...,N_p\}
\end{equation}
where $\textbf{x}_i(t)$ stands for the position of particle $i$ at time $t$.\\
Thus, a precise microscopic description of the particle dispersion in the sample is provided and macroscopic features as the macroscopic diffusion coefficient $\mathcal{D}$ can be estimated for various flow conditions.\\
Unfortunately, solving equation (\ref{eq:stokes}) and simulating a large number of particles flowing through the sample is costly in term of computational resources. Hence, it is hardly thinkable to use DNS methods for obtaining efficient and general macroscopic descriptions of diffusion in a wide variety of porous media.\\
\textit{Introduction to probabilistic approach}\\
This issue motivated a probabilistic approach of the microscopic particles dynamic with the help of stochastic processes like the historical success of the Brownian motion modeling by \citet{Einstein1906}. 
In his work, Einstein predicted accurately the displacement probability of a pollen grain over time without considering a complex deterministic model that would have included an exhaustive description of particle to particle interactions.\\

%Mathematical tools for stochastic processes
\section{Mathematical tools for stochastic modeling}
In a general way, a stochastic process can be seen as the existence of a certain time dependent random variable $\textbf{X}(t)$ in a system (c.f. \citet[Sec. 3.1]{Gardiner1996}). The probability of $\textbf{X}(t)$ visiting successively $N_x$ states $\{\textbf{x}_i,t_i\}_{i=1}^{N_x}$ knowing the $N_y$ previously visited states $\{\textbf{y}_j,\tau_j\}_{j=1}^{N_y}$  can be expressed as
\begin{equation}\label{eq:cond_prob}
p(\{\textbf{x}_i,t_i\}_{i=1}^{N_x}|\{\textbf{y}_j,\tau_j\}_{j=1}^{N_y})\quad\textrm{where}\quad \tau_1<\tau_2<...<\tau_{N_y}<t_1<t_2<...<t_{N_x}.
\end{equation}
where $p(a|b)$ stands for the probability of $a$ if $b$ happened.
Most stochastic processes are called separable, i.e. the probability (\ref{eq:cond_prob}) defines entirely the evolution of $\textbf{X}(t)$.\\
\textit{Markov process}\\
The most famous stochastic processes in physics are the Markov processes. They are separable and a have conditional probability \eqref{eq:cond_prob} that depends only on the current state $(\textbf{x},t)$ and not on the previously visited ones, i.e. 
\[
p(\{\textbf{x}_i,t_i\}_{i=1}^{N_x}|\{\textbf{y}_j,\tau_j\}_{j=1}^{N_y})=p(\{\textbf{x}_i,t_i\}_{i=1}^N|\textbf{y}_{N_y},\tau_{N_y})
\]
where $(\textbf{y}_{N_y},\tau_{N_y})=(\textbf{x},t)$ is the current state.
This property can be seen as if the modeled system does not keep the previous events that occurred in memory which is a reasonable statement for numerous physical systems like flowing particles.\\
\subsubsection{Stochastic differential equations and probability distribution functions}
The evolution of a time dependent, one dimensional, random variable $X(t)$ that follows a Markov process can be expressed as a general stochastic differential equation (SDE) of the form
\begin{equation}
	dx = a(x,t)dt + b(x,t)\xi(t)dt
	\label{eq:langevin}
\end{equation}
called the Langevin equation (\citet[Eq. 4.1.1]{Gardiner1996}). 
On the right hand side, $a(x,t)$ and $b(x,t)$ are drift and diffusion functions, respectively and $\xi(t)$ is an zero mean, unit standard deviation uncorrelated Gaussian variable, i.e.
\[
p(\xi(t)=x)=\frac{1}{\sqrt{2\pi}\sigma}\exp(-\frac{(x-x_0)^2}{2\sigma^2})
\]
with the mean $x_0=0$ and the standard deviation $\sigma=1$.\\
The simplest case of Markov process is the pure random walk (RW)
\begin{equation}\label{eq:RW}
	dx = \xi dt \equiv dW
\end{equation}
where $dW\equiv\xi(t)dt$.
If we add a drift $\mu$ to (\ref{eq:RW}), we obtain the Wiener process
\begin{equation}\label{eq:wiener}
dx = \mu dt + dW
\end{equation}
that can be seen as an oriented random walk where $\mu$ set the direction and the average displacement speed.\\
The use of an $x$ dependent drift leads to a commonly used Markov process, the Ornstein-Uhlenbeck (OU) process (\citet[Sec. 3.8.4]{Gardiner1996})
\begin{equation}
dx = -kxdt + \sqrt{D}dW.
\label{eq:OU}
\end{equation}
which can be identified as a Langevin equation (\ref{eq:langevin}) with $a(x,t)=a(x)=kx$, $b(x,t)=\sqrt{D}=\mathrm{const}$. We can prospect that the sign of $k$ will either pushes the process to converge to a pure random walk ($k>0$) or makes it diverge ($k<0$).\\
With the help of Ito's formula (see \citet[Sec. 4.3.4]{Gardiner1996}), one can obtain the differential equation that govern the evolution of the probability density function (PDF) $p(x,t)$ for any process (\ref{eq:langevin}),
\begin{equation}
\partial_t p(x,t) = \partial_x[a(x,t)p(x,t)] + \frac{1}{2}\partial^2_x [b(x,t)p(x,t)]
\label{eq:fokkerplanck} 
\end{equation}
called the Fokker-Planck equation. Equation (\ref{eq:fokkerplanck}) can be analytically solved for each processes seen before, hence leading to analytical equation mean and standard deviation of the random variable. In the case of the OU process (\ref{eq:OU}), an explicit calculation (\cite[Sec. 3.8.4]{Gardiner1996}) leads to mean value and standard deviation
\[
\langle X(t)\rangle=x_0 \exp(-kt) \quad\textrm{and}\quad \langle[X(t)-\langle X(t)\rangle]^2\rangle=\frac{D}{2k}[1-\exp(-2kt)]
\]
where one can see that as $t\rightarrow\infty$ the OU process tends to a steady Gaussian PDF of $0$ mean and $\frac{D}{2k}$ standard deviation which gives insights of its utility for velocity modeling in diffusion. \\
In fact, equation (\ref{eq:fokkerplanck}) can be generalized for multi-dimensional random variables $\textbf{Z}(t)\equiv(z_1(t),z_2(t),...,z_d(t))^T$ as (\citet[Sec. 3.5.2]{Gardiner1996}) 
\begin{equation}
\frac{\partial p(\textbf{z},t|\textbf{y},t')}{\partial t} =
-\sum_{i=1}^d\frac{\partial}{\partial z_i}[A_i(\textbf{z},t)p(\textbf{z},t|\textbf{y},t')]
+ \frac{1}{2}\sum_{i,j=1}^d\frac{\partial^2 }{\partial z_i \partial z_j}[B_{ij}(\textbf{z},t)p(\textbf{z},t|\textbf{y},t')]
\label{eq:mdfp}
\end{equation}
where $t$ and $t'$ denote two different times, $\textbf{y}$ and $\textbf{z}$ two different values of the random vector $\textbf{Z}$.\\ 
We can see now that drift and diffusion are expressed in a general manner by the vector $\textbf{A}(\textbf{z},t)$ and the matrix $\underline{\textbf{B}}(\textbf{z},t)$, respectively.\\
Thus, one of the major goals of actual research in porous media is to set the functions $\textbf{A}(\textbf{z},t)$ and $\underline{\textbf{B}}(\textbf{z},t)$ using geometry based arguments (e.g. pore length $L$ and porosity $\phi$) and macroscopic features of the flow (e.g mean velocity $U$ and pressure gap $\Delta\textrm{P}$) in order to build a stochastic model that could reproduce non Fickian diffusion dynamic and transition to Fickian regime.\\

\section{Estimation of PDF by sampling in particle flow}
In the case of flowing particles, modeling the velocity of a particle $i$ as a Markovian variable $\textbf{U}_i$ and knowing a state $(\textbf{x}_0,\textbf{u}_0,t_0)$ would entirely define its position time serie 
\[\textbf{X}_i(t)=\int_{t_0}^{t}\textbf{U}_i(t')dt'\]
which is an equivalent expression of (\ref{eq:traject}).
Hence, one needs to estimate Lagrangian velocity PDF $p(\textbf{u},t)$ from empirical data for example, the parameters $D$ and $k$ of \eqref{eq:OU}. 
\\The PDF of a random variable $Q(t)$ can be exactly defined by sampling over infinite $N_p$ particle
\[
p(Q(t)=q)\equiv p(q,t)=\lim\limits_{N_p\rightarrow\infty}p_{N_p}(q,t) \quad\textrm{where}\quad p_{N_p}(q,t)=\frac{1}{N_p}\sum_{i=1}^{N_p}\delta(q-q_i(t))
\]
It is worth noting that the approximated PDF $p_{N_p}(q,t)$ can be obtained easily thanks to DNS data $\{q(\textbf{a}_i,t)\}_{i=1}^{N_p}$ for $N_p$ particles, differentiated here by their respective initial position $\{\textbf{a}_i\}_{i=1}^{N_p}$.\\

However, in order to provide an accurate estimation of $p(q,t)$ with a finite number of particles, the DNS data 
must be sampled during a sufficiently long simulation time $T_{sim}$ and a for a sufficient number of particle. 
Mathematically, these sufficiency conditions should be understood as ergodicity on the particle data series, i.e.
\begin{equation}
\begin{array}{c}
	\langle q(\textbf{a}_i,t)\rangle_t = \langle q(\textbf{a},t_n)\rangle_{\textbf{a}}\\
	\Leftrightarrow\\
	\int_{0}^{\infty}q(\textbf{a}_i,t) p(q(\textbf{a}_i,t),t)dt=\int_{-\infty}^{\infty}q(\textbf{a},t_n)p(q(\textbf{a},t_n),t_n)d\textbf{a}
\end{array}
\label{eq:ergodicity}
\end{equation}
One can understand ergodicity condition as the equivalence between averaging over an infinite number of particles at a given time $t_n$ and averaging over an infinite time for an arbitrary particle $i$. 
More practically, ergodicity condition with a finite number of particle can still be tested for any particle $i$ at any time $t_n$ with the error
\begin{equation}\label{eq:ergo_err}
	e(\textbf{a}_i,t_n)=|\frac{1}{N_T}\sum_{k=1}^{N_T}q(\textbf{a}_i,t_k)-\frac{1}{N_p}\sum_{j=1}^{N_p}q(\textbf{a}_j,t_n)|.
\end{equation}
Thus, ergodicity is the most important condition to respect in order to estimate accurate PDF with DNS data. The Markov model parameters can then be set to fit the DNS data PDF with random variables PDF under ergodic condition.\\
The whole procedure, described above, forms the basis of the studied papers in the following literature review.\\

\section{Literature review and recent results}
\subsection{Puyguiraud et al. 2019}
In their latest work, \citet{Puyguiraud2019} presented an analysis of Lagrangian pore-scale velocity series in a three-dimensional sample of a Berea sandstone, well known for retaining oil and natural gas.
Their numerical porous domain was a cube of $0.95$mm$^3$, pore length $L_0=0.15$mm and tortuosity $\kappa=1.75$, obtained with a scanning resolution of $300^3$ voxels. 
With the help of the SIMPLE algorithm of OpenFoam and a mesh of $900^3$ cubes that fitted perfectly the scan voxels, the single-phase pore-scale flow has been obtained by solving (\ref{eq:stokes}) with additional non slip boundary conditions on the lateral sides of the domain and at solid interface, i.e.
\begin{equation}\label{eq:nonslip_sideBC}
\textbf{u}(\textbf{x})=0 \quad\forall \textbf{x}\in \{\textbf{y}\in\Omega_d|y_2,y_3\notin]0,L_0[\}\cup\{\textbf{y}\in\Omega_d|\rho_d(\textbf{y})=1\}.
\end{equation}
Their resulting mean flow velocity, $U=8.05\cdot 10^{-4}$m/s, combined with the characteristic pore length, $L=1.5\cdot10^{-4}$m, gives an approximated Péclet number\footnote{using $D_m\sim10^{-5}$cm$^2$/s corresponding to an average value of molecular diffusion in liquid water at $25^\circ$C} $\mathrm{Pe}\sim100$. \\
Then, \citeauthor{Puyguiraud2019} investigated the evolution of $N_p=10^6$ particles injected as a uniform or flux-weighted distribution on an initial particle domain $\Omega_0$, i.e.
\begin{equation}\label{eq:unif_distr}
\rho_u(\textbf{a})=\frac{\mathcal{I}(\textbf{a}\in\Omega_0)}{\int_{\Omega_0} d\textbf{a}}
\end{equation}
and
\begin{equation}\label{eq:flux_weighted}
\rho_{fw}(\textbf{a})=\frac{\norm{\textbf{u}(\textbf{a})}_2\mathcal{I}(\textbf{a}\in\Omega_0)}{\int_{\Omega_0} \norm{\textbf{u}(\textbf{a})}_2 d\textbf{a}}
\end{equation}
respectively, where $\mathcal{I}(a)=1$ if $a$ is true and $0$ otherwise. 
The trajectories equation (\ref{eq:traject}) was solved for each particles using a quadratic interpolation of the face velocities.
Particles that reached the outlet plane with a speed $u_{L_0}\equiv\norm{\textbf{u}[(L_0,x_2,x_3)^T]}$ were reinjected randomly among the set of position on the inlet plane
\[
\Omega_{in}=\{\textbf{x}\in\Omega_d|\,x_1=0,\,\norm{\textbf{u}(\textbf{x})}_2\in[u_{L_0}\pm\Delta u]\}\quad\rm{with}\quad \Delta u=\frac{u_{L_0}}{200}.
\]
During the trajectories computation, two different Lagrangian velocity statistics have been sampled from each particle: one sampled with constant time intervals, i.e. isochronous t-Lagrangian velocity statistics
\[
\{v(\textbf{a},t_{n})\}_{n=0}^{N_T}\quad\textrm{with $\{t_n\}_{n=0}^{N_T}$ s.t.}\quad \Delta t_n\equiv t_{n+1}-t_n = \Delta t=\mathrm{const},
\] 
the other sampled with constant space intervals, i.e. equidistant s-Lagrangian velocity statistics
\[\{v(\textbf{a},s_{n})\}_{n=0}^{N_S}\quad\textrm{with $\{t_n\}_{n=0}^{N_S}$ s.t.}\quad \Delta s \equiv\norm{\textbf{u}(t_n)}_2\Delta t_n = \mathrm{const}.
\]
After a graphical comparison between the s- and t-Lagrangian velocity series (see \citet[Fig. 2]{Puyguiraud2019}), intermittency in the isochronous data convinced \citeauthor{Puyguiraud2019} to model uniquely the equidistant velocity series.\\
They considered their s-Lagrangian velocity serie as a stationary ergodic Markov process. 
To demonstrate the correctness of this hypothesis, they modeled the dynamic of the s-Lagrangian PDF $p(v,s)$ with a process that followed empirical, discrete transition probabilities $T_{nm}\equiv r(v_m,s_m-s_n|v_n)$ obtained from the DNS data as
\[
T_{nm}=\frac{1}{N_p}\sum_{i=1}^{N_p}\frac{1}{N_S}\sum_{k=0}^{N_S}\mathcal{I}[v_n \leq v_i(s_n) < v_n + \Delta v_n]\cdot\mathcal{I}[v_m \leq v_i(s_n+\Delta s) < v_m + \Delta v_m]
\]
where $v_i(s_n)\equiv v(\textbf{a}_i,s_n)$ and $\{\Delta v_l\}_{l=1}^{N_v}$ represent a division of the velocity range in $N_v$ bins s.t. $\min[v(\textbf{a},s)]=v_0$ and $\max[v(\textbf{a},s)]=v_{0}+\sum_{l=1}^{N_v}\Delta v_l$.\\
The empirical transition probability matrix $T_{nm}$ showed notable variations for $\Delta s\in[\frac{1}{150}L_0,\frac{8}{3}L_0]$ as displayed on \citet[Fig. 7]{Puyguiraud2019}. 
This has been interpreted as a consequence of the non exponential character of the Lagrangian velocity correlation function for distances smaller than $L_0$ which motivated \citeauthor{Puyguiraud2019} to set the equidistant sampling rate at $\Delta s = 8/3 L_0$.\\
The dynamic of the s-Lagrangian velocity PDF resulting from $10^8$ empirical Markov processes reproduced accurately the evolution of the same PDF for $10^7$ particles tracked along $3.8L_0$, i.e $\frac{3.8L_0}{\kappa U}\approx 10^{-1}T$, initialized with a flux weighted distribution.\\
Thereafter, they relaxed their fully empirical Markov model by using a Bernoulli process 
\begin{equation}\label{eq:bernoulli}
r(v,\Delta s|v')=P(v) + [1 - P(v)] e^{-\frac{\Delta s}{l_c}},
\end{equation}
where $l_c$ is a cutting length parameter. This model did not reproduced the evolution of $p(v,s)$ over time as the empirical one but \citeauthor{Puyguiraud2019} used it to set the cutting length $l_c=2.5l_p$ by adjusting the convergence of the mean velocity with DNS data for large $s$. \\
The identified cutting length has been then used to construct an OU process $w(s)$
\[
dw = -l_c^{-1}wds + \sqrt{2l_c^{-1}}\xi(s)ds
\]
solved numerically for an ensemble of $10^7$ particles using the explicit Euler scheme 
\[
w_{n+1}=w_n-l_C^{-1}w_n\Delta s + \sqrt{2l_c^{-1}\Delta s}\xi_n
\]
where $f_n$ denotes $f(n\Delta s)$.
The value $w(s)$ has been linked to the velocity $v(s)$ thanks a Smirnov transform that enforced the equality between the cumulative distributions of $w$ and $v$, $\Pi(v)$ and $\Phi(w)$ respectively. 
Hence, the next step velocity of particle $i$, $v_{n+1}^i\equiv v(\textbf{a}_i,(n+1)\Delta s)$, can be obtained from $w_{n+1}$ as
\[
\begin{array}{rrcl}
\Pi(v_{n+1}) = \Phi(w_{n+1}) \Leftrightarrow& \int_0^{v_{n+1}} P(v')dv' &=& \int_{-\infty}^{w_{n+1}}\Phi(w')dw'\\
\Rightarrow& \frac{1}{n+1}\sum_{m=0}^{n+1}v^i_m&=&\frac{1}{n+1}\sum_{m=0}^{n+1}w^i_m\\
\Rightarrow & v^i_{n+1} &=& \sum_{m=0}^{n+1}w^i_m - \sum_{m=0}^{n}v^i_m.
\end{array}
\]
With this coupled OU model, \citeauthor{Puyguiraud2019} succeeded to model the dynamic of the ensemble s-Lagrangian velocity PDF for uniform and flux weighted initial particle distributions.
To conclude their work, they affirmed that this latter model has been parameterized using hydraulic and geometric features of the pore media. 
However, it would interesting to know more about the sensitiveness of the OU process with respect to the cutting length since a geometric argumentation of $l_c=2.5L_0$ has not been provided.
The versatility of their best stochastic model could have been tested for other porous media as well.

\subsection{Dentz et al. 2017}
On their side, \citet{Dentz2017} studied a synthetic porous media built by simulating sedimentation of irregularly shaped grains generated with an empirically fitted Weibull distribution
\begin{equation}\label{eq:weibull}
	p_W(x)=bkx^{k-1}\exp(bx^k)
\end{equation}
with parameters $b$ and $k$.\\
They obtained a cubic sample of length $L=2$mm, porosity $\phi=0.35$, pore length $L_0=0.28$mm, tortuosity $\kappa=1.6$ and applied a pressure gradient at the boundaries that imposed a mean flow velocity $U=5.73\cdot10^{-6}$m/s. By changing the value of the microscopic diffusion coefficient $D$, they analyzed their sample on a range of Péclet number $Pe\in\{30, 3\cdot10^2, 5\cdot10^2, 10^3\}$.\\
As said in the beginning of this chapter, \citeauthor{Dentz2017} demonstrated the unavailability of \eqref{eq:adv_diff} in complex porous medium by comparison of the analytical solution with a pore-scale Eulerian explicit DNS of \eqref{eq:stokes} and the local transport PDE
\begin{equation}\label{eq:micro_adv_diff}
\pdv{t}c(\textbf{x},t)+\nabla\cdot[c(\textbf{x},t)\textbf{u}(\textbf{x},t)]-D \nabla^2c(\textbf{x},t)=0.
\end{equation}
As comparison criterion, they defined the breakthrough curves
\[
f(x,t)=1-\frac{\int_0^L\int_0^L\,c(x_1=x, x_2,x_3,t)u_1(x_1,=x,x_2,x_3)dx_2\,dx_3}{\int_0^L\int_0^L\,u_1(x_1=x,x_2,x_3)dx_2\,dx_3}.
\]
that expresses the particle-free fraction of the outlet plane setting $x=L$.\\
They demonstrated that the macroscopic model \eqref{eq:adv_diff} was not able capture the tailing behavior of the pore-scale description at intermediate and long time ($t/T\approx 10^0$ and $\approx 10^2$, respectively).\\
Thereafter, \citeauthor{Dentz2017} provided a physical model of the Eulerian velocity magnitude PDF $p_e(v\equiv\norm{\textbf{u}(\textbf{x})}_2)$  obtained from the DNS data. 
They modeled the PDF as sum of pore contributions, assumed as local parabolic Poiseuille flows 
\begin{equation}\label{eq:dentz_poiseuille}
v(r)=\left(\frac{d_p^2-(2r)^2}{d_0^2}\right)v_0
\end{equation}
where $d_p$ is the pore diameter, $v_0$ and $d_0$ are a characteristic velocity and diameter, respectively.
They assumed that the pore diameter distribution $p_d(d)$ was similar to their the grain size distribution, i.e. a Weibull distribution \eqref{eq:weibull} with parameters $k=\alpha$ and $b=(d_0)^{-\alpha}$
\[
p_d(d)=\frac{\alpha}{d_0}\left(\frac{d}{d_0}\right)^{\alpha-1}\exp[-(d/d_0)^\alpha].
\]
Hence, they achieved to obtain an analytical expression for the Eulerian velocity PDF :
\begin{equation}\label{eq:dentz_pev}
p_e(v)=\frac{\exp[-(v/v_0)^{\alpha/2}]}{v_0\Gamma(1+\alpha/2)}
\end{equation}
with $\alpha\approx2.5$ and $v_0\approx2.1U$ estimated by fitting with the DNS Eulerian velocity PDF.
\citeauthor{Dentz2017} also estimated the mean pore velocity distribution $p_m(v)$ in a similar way and approximate it with a gamma distribution
\begin{equation}\label{eq:dentz_pmv}
p_m(v)=\frac{\beta}{U}\left(\frac{\beta v}{U}\right)^{\alpha/2}\exp\left(-\frac{\beta v}{U}\right) 
\end{equation}
with $\beta=(\alpha/2+1)$. 
It is worth noting that their model of $p_m(v)$ showed good agreement with empirical data by using a unique parameter.\\
After this analytical modeling, \citeauthor{Dentz2017} continued their investigation by using a particle based simulation ($N_p=10^9$) of the transport in their synthetic porous domain for simulation time up to $t/T=10^2$ and tried to model the resulting breakthrough curve with incrementally complex time domain random walks (TDRW).\\
Their first model considered purely advective processes, $\mathrm{Pe}=\infty$, with equidistant space transitions by defining transition time as $\Delta t_n=l/v(t_n)$ with a transition length  $l\sim L_0$. 
They estimated analytically the asymptotic behavior of the breakthrough curve $f(x,t\gg T)\propto t^{-2}$ for the purely advective case by approximating the transition times PDF as a flux weighted Eulerian velocity PDF \eqref{eq:dentz_pev}.\\
%\[\psi_a(t)=\frac{l}{t^3U}p_e(l/t)\]
The predicted $t^{-2}$ dependence of the breakthrough curve for purely advective transport has been confirmed by the particle simulation for any Péclet value Pe$\in\{30, 3\cdot10^2, 5\cdot10^2, 10^3\}$. However this TDRW model did not capture the breakthrough curve tailing at either intermediate nor long time.\\
In the second model, diffusion has been added by considering the Lagrangian velocity PDF as a flux weighted pore mean velocity PDF \eqref{eq:dentz_pmv}. This hypothesis has been motivated by the fact that particles should experience a pore-scale Poiseuille flow \eqref{eq:dentz_poiseuille} that is smoothed by diffusion.
The direction of spatial transition has also been randomized to allow upstream transitions with probability 
\[w_u(v)=\frac{D\tau_v}{l^2}.\]
This advective-diffusive TDRW model showed breakthrough curve in good agreement with DNS data for the studied Péclet range up to long time where it failed to capture the tailing behavior. \\
To catch the behavior of the DNS breakthrough curve at long time, \citeauthor{Dentz2017} changed their advective-diffusive TDRW by using an arbitrary small transition length $\Delta s\ll L$. 
They also replaced the pore-scale features used in the previously estimated PDF, i.e. pore-scale velocity $v$ and diffusion coefficient $D$, by macroscopic ones, i.e. mean Eulerian velocity $U$ and hydrodynamic coefficient $\mathcal{D}$.\\
Moreover, they modeled the probability for a particle to be trapped in low velocity region with a Poisson distribution. 
The parameters of the process has been estimated from the DNS breakthrough curve at long time resulting in $\gamma\approx1.25\cdot10^{-4}\tau_D$ and $\langle\tau_f\rangle\approx\frac{1}{2}\tau_D$ for the rate and the mean trapping time, respectively.\\
By combination of the diffusive and the previous macroscopic trapping TDRWs, \citeauthor{Dentz2017} succeeded in reproducing the complete behavior of the DNS breakthrough curve with a stochastic model for $Pe\in\{30, 3\cdot10^2, 5\cdot10^2, 10^3\}$. This final model contained five main parameters  
\[v_0=2.1U,\quad\alpha=2.5,\quad l(\mathrm{Pe})=f(\mathrm{Pe})L_0,\quad\gamma\tau_D=2.5\cdot 10^{-4}\quad\textrm{and}\quad\langle\tau_f\rangle/\tau_D=0.5\]
that can be counted as nine if one take into account that the step length $l$ has been slightly varied for each Péclet value ($f(\mathrm{Pe}=[30, 3\cdot10^2, 5\cdot10^2, 10^3])=[0.7,0.77,0.77,1]$) without any explicit explanation.\\
Like \citeauthor{Puyguiraud2019}, \citeauthor{Dentz2017} focused their work on equidistant sampled Lagrangian velocity magnitude and for a unique porous sample. However, the stochastic models has been tested for different flow features of the dispersion mechanism. 
One can see the breakthrough curve of \citeauthor{Dentz2017} as a more complex feature than the dynamic of Lagrangian PDF of \citeauthor{Puyguiraud2019} since it required to model accurately the time spent in slow velocity region.
It would be interesting to measure the breakthrough curve resulting from the OU process of \citet{Puyguiraud2019}.
It can be expected that the TDRW would show better performances than the OU process since the former required to set five parameters and the latter only one.

\subsection{Meyer et al. 2016}
It is interesting to relate both previously cited papers to the work of \citet{Meyer2016} who focused on isochronous time Lagrangian velocity PDF obtained from six different physical samples. \\
Their main goal was to test the ability of a previously developed Markov process to universally describe transport in porous domain.
To do so, they studied different physical porous samples with various geometric features, from simple bead pack structure with $\kappa_{\rm{BP}} = 1.26$, $L_{\rm{BP}}=0.6$mm and $\phi_{\rm{BP}}=35.9\%$, to complex Estaillades structure characterized by $\kappa_{\rm{Est}} = 2.26$, $L_{\rm{Est}}=3.31$mm and $\phi_{\rm{Est}}=10.9\%$.
The flow inside each porous medium has been solved for a pressure gap $\Delta\textrm{P}=1$Pa. 
In presence of thermal diffusion, the Péclet numbers could be approximated as $\rm{Pe}\sim\frac{UL}{D_m}$ which would give a range of $2\cdot10^2\leq\rm{Pe}\leq2\cdot10^4$ that is a little bit higher than the Péclet range studied by \citet{Dentz2017}.\\
\citeauthor{Meyer2016} then modeled the Lagrangian velocity PDF extracted from isochronous time series as skew-normal PDF regarding the log velocity magnitude, i.e.
\begin{equation}\label{eq:meyer_skew}
p(v)=\frac{1}{\sqrt{2\pi}\sigma}\exp\left[-\frac{(v-\mu)^2}{2\sigma^2}\right](1-\erf\left[-\frac{\alpha(v-\mu)}{\sqrt{2}\sigma}\right])
\end{equation}
where $\alpha$ is the skewness parameter and $v=v(t)\equiv\ln[\norm{\textbf{u}(t)}_2/U]$ . This PDF enabled to build a stochastic description of particle velocity evolution using a Markov process of the form \eqref{eq:langevin} with 
\[b^2(v,t)=d(v)=c\exp[bv]\]
and 
\[a(v,t)=a(v)=\frac{d(v)}{2}\ln[p(v)d(v)]\]
resulting in a five parameters modeling ($b,c,\alpha,\mu,\sigma$) for the Lagrangian velocity magnitude.
They also provided an analytical model for directional angles $\theta=\cos^{-1}(\frac{\textbf{u}\cdot\textbf{U}}{\norm{\textbf{u}}_2\norm{\textbf{U}}_2})$ as
\[p(|\theta|)=\frac{|\theta|}{\sigma_\theta^2}\exp\left[-\frac{|\theta|}{2\sigma_\theta^2}\right]\]
where the relation between $\sigma_\theta$ and the tortuosity, i.e. $\kappa=1/\langle cos(|\theta|)\rangle$, was confirmed by the DNS data. This PDF is then linked to an oscillatory stochastic process with the help of a Smirnov transform used in the same way as in \citet{Puyguiraud2019}. \\
Thanks to this analysis, \citeauthor{Meyer2016} showed that the traveled distance $L_\theta$ with constant $\theta$ sign could be efficiently represented by an exponential PDF for all media considered (c.f. \citet[Fig- 12]{Meyer2016}). They also modeled a second directional angle $\beta$ with an OU process, identified by analyzing the DNS data.\\
With the help of the five parameters Markov modeling of velocity magnitude combined with the oscillatory and OU stochastic models for the directional angles, \citet{Meyer2016} were able to generate three-dimensional trajectories that could be confused with DNS ones. \\
More quantitatively, they tested the performance of the stochastic model by measuring the particle position PDF over time and its variance for $2\cdot10^5$ particles in both longitudinal and transversal directions.
It appeared that the predicted dispersion was more accurate for medium with low tortuosity and high porosity samples. 
\cite{Meyer2016} explained it by the fact that samples of complex media need to be larger in order to be considered as an REV.\\

It can be hard to compare the samples studied by \citeauthor{Meyer2016} with the Berea sandstone of \citet{Dentz2017} and the synthetic samples of \citet{Puyguiraud2019}. 
The Bentheimer structure of \citet{Meyer2016} is surely the most similar to both media in term of tortuosity with $\kappa_{\rm{Bent}}=1.65$. 
However, its porosity does not match ($\phi_{\rm{Bent}}=21.1\%$ against $\phi_{\rm{Berea}}=35\%$).
In any case, the comparison between various media done in \citet{Meyer2016} tended to show that structures with low porosity and high tortuosity were the most complex to model. 
It would be interesting to try to extract and test an approximated law that links the model parameters to the geometric features of the medium.\\

In a more general point of view, the three presented papers showed three different manners to test the results obtained from stochastic modeling. 
The Lagrangian velocity PDF rules the displacement of the particles, hence it seems clear that the model should recover the behavior of this distribution at first.
The question of equidistant and isochronous sampling stays open but isochronous series displayed a more complex behavior in both \citet{Meyer2016} and \citet{Puyguiraud2019} works which may indicate that it contains, generally, more information about the structure of the flow.\\
The comparison between particle breakthrough curve done by \citet{Dentz2017} over time is a very good way to obtain condensed results without losing much information. However, it becomes useless as soon as the considered plane is saturated.\\
The comparison criteria chosen by \citet{Meyer2016} are more exhaustive as it shows time evolution of particle position PDF but may be harder to read.\\

\section{Report plan and attack angle}
In this work, the transition between advective and diffusive regimes in porous media is investigated by analyzing isochronous time series obtained from direct numerical simulations. 
The critical diffusion coefficient as well as the size of the transient area are evaluated for \textbf{two different sample :} Bentheimer stone and \textbf{????}.
The results are then explained with a physical reasoning on the macroscopic structure of the different sandstone which leads to some conclusion about the importance of molecular diffusion for transport in porous media.



%Weitere Kapitel

%2. Vorgehen
%------------------------------------------------------------------------------------------------------------------------------
\chapter{Methodology and validation}
\section{Simulation tool and data processing}
Taken as a tool, the direct numerical simulation program \texttt{streamlinesnt3D}, written in \textsc{FORTRAN} by \citet{Meyer2017}, is the angular stone of this work's methodology. 
Briefly, the program must be fed with a voxel mesh representing the solid regions as well as an Eulerian velocity field corresponding to a numerical solution of the Stokes equation \eqref{eq:stokes} computed beforehand.
The studied voxel meshes have been obtained from the Digital Rocks Portal website (\citet{digitalrocksportal}) and the Eulerian flow were computed for a unitary pressure gradient as boundary condition.\\
\texttt{streamlinesnt3D} can run in two different modes.
The first mode simulates the evolution of a single particle during a user provided time $T_{max}$ or until the particle completed a certain number of passage through the sample denoted by SLSM (stream line section max). 
The particle position and velocity are sampled given a user provided cell storage frequency (CSF) where $\mathrm{CSF}=N$ will sample the particle state every $N$ time steps.\\
The second mode is meant to track simultaneously $N_p$ particles forming a plume. The output data are snapshots regarding a user provided timetable. 
In opposite to the single particle tracker, this mode does not allow to display trajectories but is more efficient to evaluate macroscopic mass transport since it can run in parallel with OpenMP.

Finally, both program modes use the same tracking method that interpolates the displacement of a particle from a face to another taking advection and diffusion into account.
On one hand, advection is simulated by determining, with a linear approximation of the velocity gradient, which next cell face a particle will reach and how long it will take (c.f. \cite{Pollock1988}).
On the other hand, diffusive behavior has been modeled as a probability for the particle to deviate from the velocity field in a random direction with a velocity magnitude related to the molecular diffusion $D_m$ (c.f. \citet{Meyer2017}).\\
The key of combining advection and diffusion in this algorithm resides in setting the computation time step as the minimum between the advection only time step $dt_A$ from \citet{Pollock1988} and a diffusive time step $dt_\zeta:=2\times\frac{L_c^2}{D_m}$ with $L_c$ the cell length.
Thus, a large diffusion coefficient will lead to a small diffusive time step 

As well as every classical Eulerian CFD numerical scheme, the program must fulfill the Courant–Friedrichs–Lewy (CFL) condition \eqref{eq:CFL} in order to be stable. 
\begin{equation}
	\Delta t < C\frac{\textbf{u}_i\cdot \hat{\textbf{x}}_d}{\Delta x} \quad \forall i\in\{1,2,...,N_p\},\,\forall \quad \hat{\textbf{x}}_1\in\{\hat{\textbf{x}}_2,\hat{\textbf{x}}_3,\hat{\textbf{x}}_d\}.
	\label{eq:CFL}
\end{equation}
In order to minimize the computation time, the program set the time step to the maximal value that fulfills the CFL condition for the current particle state. 
The sampled data form thus an heterogeneous time series that should be post processed in order to obtain results that can be compared to \citet{Meyer2016}.

\begin{algorithm}
	\caption{Isochronous transform post processing pseudo-code }
	\label{alg:isochronous}
	\begin{algorithmic}
		\State \textbf{Input :} $\{(t_i,v_i)\}_{i=1}^N$
		\Comment{Heterogeneous time series}
		\State \textbf{Output :} $\{(\hat{t}_j,\hat{v}_j)\}_{j=1}^{\hat{N}}$
		\Comment{Isochronous time series}
		\State $\hat{\Delta t} \gets \frac{1}{N}\sum_{n=0}^{N-1}(t_{n+1}-t_{n})$
		\Comment{Mean DNS time step}
		\State $n \gets 1$, $j \gets 1$, $\hat{t} \gets t_1$
		\Comment{Temporal loop initialization}
		\While {$n \leq N$}
		\While {$t_n <= \hat{t} <= t_{n+1}$}
		\State $\hat{v}_j \gets v_n + \frac{\hat{t}-t_n}{t_{n+1}-t_n} v_{n+1}$
		\State $\hat{t}\gets \hat{t} + \hat{\Delta t}$
		\State $j \gets j + 1$
		\EndWhile
		\State $n\gets n+1$
		\EndWhile
	\end{algorithmic}
\end{algorithm}

Hence, a post processing script has been implemented in \textsc{Python} using \textsc{Scipy} packages (\cite{scipy}) to transform the rough output of the program, denoted DNS, into the isochronous time series needed for the statistical analysis.
Algorithm \ref{alg:isochronous} shows the method used to perform the isochronous transformation of the DNS data which consists in using equidistantly linear interpolation.
It is worth noting that the average DNS time step has been chosen as the equidistant time interval since it enables to keep the length of the resulting isochronous series comparable to the original one.\\
\begin{figure}
	\centering
	\includegraphics[scale=0.8]{Figures/check_iso_transf.pdf}
	\caption{Example of isochronous transform for $100$ DNS points in the Bentheimer 1000 sample. As a reminder, the log velocity magnitude is denoted $v=\ln{(u/U)}$ where $u:=||\textbf{u}||_2$ is the Lagrangian velocity magnitude}
	\label{fig:checkisotransf}
\end{figure}
Figure \ref{fig:checkisotransf} displays the results of Algorithm \ref{alg:isochronous} for three independent simulations.
It can be observed that oscillations with period smaller than one half of the average time step cannot be sampled by the isochronous series (see Nyquist–Shannon sampling theorem) which makes this process similar to a low pass filter in signal processing.
Fortunately, the statistics presented previously should not be affected by such loss of information.\\

This method has been used in a similar way to compute equidistant time series as for the directional angle, $\theta (t)\in[-\pi,\pi]$, which indicates the orientation of the particle velocity vector, $\textbf{u}_n :=\textbf{u}(t_n)$, with respect to the mean flow direction $\textbf{U}$.
More precisely, the angle $\theta$, defined as
\begin{equation}
\cos \theta_n = S_{n} \frac{\textbf{u}_n}{||\textbf{u}_n||_2 }\cdot \frac{\textbf{U}}{||\textbf{U}||_2},
\end{equation}
where $S_{n,n-1}\in\{-1,1\}$ is a factor that changes its sign only if the velocity direction went across the plane spanned by $\textbf{U}$ and $\textbf{U}\cross\textbf{u}_{n-1}$ on the previous step (c.f. Fig. \ref{fig:anglemeasuring}), i.e.
\begin{equation}
S_{n} = \mathrm{sign}_+(\textbf{u}_n\cdot [(\textbf{U}\cross\textbf{u}_{n-1})\cross \textbf{U}])\cdot S_{n-1} 
\end{equation}
with sign$_+(x\geq 0)=1$ and sign$_+(x< 0)=-1$.


\begin{equation}
\sin\Delta\beta = \frac{||[\textbf{U}\cross \textbf{u}(t+dt)]\cross[\textbf{U}\cross \textbf{u}(t)]||_2}{||\textbf{U}\cross \textbf{u}(t+dt)||_2 \cdot||\textbf{U}\cross \textbf{u}(t)||_2}
\end{equation}

\begin{figure}
	\centering
	\includegraphics[width=0.4\linewidth]{Figures/angle_measuring}
	\caption{Illustration of the directional angle $\theta$. In the case where $\textbf{u}_{n+1} = \textbf{u}_a$, $S_{n+1}$ is positive since $\textbf{u}_a$ sits on the same side of the plane $\hat{n}$ as $\textbf{u}_n$. Inversely, if $\textbf{u}_{n+1} = \textbf{u}_b$, $\theta_{n+1}$ sign is opposed to $\theta_n$ sign. }
	\label{fig:anglemeasuring}
\end{figure}


\section{Validation of \cite{Meyer2016} results for advection dominated regime $(\mathrm{Pe}=\infty)$}
In order to validate the post processing algorithms, the main results of \cite{Meyer2016} have been reproduced by running \texttt{streamlinesnt3D} in particle tracking mode on the ETH Euler cluster (scicomp.ethz.ch/wiki/Euler) with zero diffusion ($D_m=0$), sampling at each step ($\mathrm{CSF}=1$) and for $2\times 10^4$ sample crossings.\\ 
Figure \ref{fig:lvm_pdf_peinf} displays the resulting log velocity magnitude PDF. 
It is worth noting that the PDF obtained from isochronous series differs drastically from the rough DNS data by its wider distribution and its slower mean velocity.
Thus, this illustrates efficiently how DNS data can lead to an underestimation on the small velocities due to the variable time step meant to accelerate the computation.\\
The Markov model designed by \cite{Meyer2016} has been implemented as well and fits perfectly the PDF.
It is interesting to compare those PDF with the distribution of the Eulerian flow of the sample (dashed line in Figure \ref{fig:lvm_pdf_peinf}).
\begin{figure}
	\centering
	\includegraphics[scale=0.8]{Figures/LVM_pdf_DM=0.pdf}
	\caption{Log velocity magnitude PDF for DNS and Isochronous data. The solid red line displays the skew normal probability distribution parametrized in \citet{Meyer2016} for Bentheimer 1000 sandstone. The dashed gray line represents the distribution of the velocity distrbution of the field among the sample. The particle has been tracked for $2\times 10^4$ passages and its statistics were measured at each time step.}
	\label{fig:lvm_pdf_peinf}
\end{figure}

In the case of plume tracking, the transport results for early time have also been retrieved as shown on Figure blabla. 
The snapshots at later time have been left aside due to a lack of computational resources. 

\begin{figure}
	\centering
	\includegraphics[scale=0.8]{Figures/Plume_vs_markov_PeInf.pdf}
	\caption{Comparison between DNS plume dispersion results for $2\times 10^5$ particles (solid) and Markov model for $10^5$ particles (dashed) in three snapshots without molecular diffusion.}
	\label{fig:lvm_pdf_peinf}
\end{figure}

$$
\hat R(k) = \frac{1}{n-k} \sum_{t=1}^{n-k} \left(\frac{X_t-\mu_Y}{\sigma_X}\right)\left(\frac{Y_t-\mu_Y}{\sigma_Y}\right) \quad\textrm{where}\quad \{Y_1,Y_2,...,Y_{n-k}\} := \{X_{k+1},X_{k+2},...,X_{n}\} 
$$

\chapter{Experiments with finite Péclet values}

\section{Plume tracking}

\section{Particle tracking}


\begin{figure}
	\centering
	\includegraphics[scale=0.8]{Figures/LVM_pdf_comparison.pdf}
	\caption{Log velocity magnitude PDF for rough DNS data and isochronous data in different diffusion regimes.}
	\label{fig:lvm_pdf_peinf}
\end{figure}

\begin{figure}
	\centering
	\includegraphics[scale=0.8]{Figures/model_test.pdf}
	\caption{Plume displacement for different Péclet values.}
	\label{fig:lvm_pdf_peinf}
\end{figure}
%Literaturverzeichnis
%------------------------------------------------------------------------------------------------------------------------------
%% Literaturverzeichnis
%*****************************************************************************************************

% Zwischen \begin{thebibliography}{99} und \end{thebibliography} kann das Literaturverzeichnis erweitert werden. 
%Der Text in der geschweiften Klammer direkt nach \bibitem ermöglicht das Zitieren mit dem \cite-Befehl.


\begin{thebibliography}{99}
\addcontentsline{toc}{chapter}{Bibliography}

\bibitem{citation_reference} \textsc {Autor,}  \emph{Buchtitel}, Jahr, Verlag.
 
\end{thebibliography}

\bibliography{../Sources/MT_bib}
\bibliographystyle{unsrtnat}
%Anhang
%------------------------------------------------------------------------------------------------------------------------------
\appendix
\renewcommand{\chaptermark}[1]{         		% Benutze "Anhang" statt "Kapitel" in den Kopfzeilen
\markboth{Anhang\ \thechapter.\ #1}{}} 

\chapter{Appendix}
\section{Particle tracking algorithm}
The particle tracking method interpolates the displacement of a particle from a face to another using a linear approximation of the velocity gradient and a probabilistic approach for diffusion.

\begin{algorithm}[H]
	\caption{Particle tracking algorithm}
	\label{alg:part_track}
	\begin{algorithmic}[1]
		\Procedure {Pollock}{$d$, $n$, $L_c$, $v$, $g$, $bnd$, $\textbf{i}_s$, $\textbf{x}_s$, $D_m$, $c_{max}$, $t_{max}$, $m$, $dt$, $dx$, $u$, $a$, $n_s$, trace, $x_c$, seed}
		\State $dt_z\leftarrow0$ \Comment Init. diffusion step time
		\State $dt_{ze}\leftarrow0$ \Comment	Init. diffusion elapsed time
		\State $\textbf{i}\leftarrow \textbf{i}_s$ \Comment Init. current cell index as starting cell
		\State $\textbf{x}_i\leftarrow \textbf{x}_s$ \Comment Init. current position as starting position
		\State $e_i\leftarrow0$ \Comment Init. exit interface ($e_i=\pm j$ means exit direction $\pm\hat{\textbf{x}}_j$)
		\State $m\leftarrow0$ \Comment Init. number of streamline cells
		\State $t\leftarrow0$ \Comment Init. current streamline travel time
		\Repeat
		\If {$D_m > 0$ and $dt_{ze} \geq dt_z$} \Comment Diffusion step
		\State $\textbf{Z}\leftarrow $ normal gaussian deviates s.t. $Z_k\neq0\,\forall k=1,d$
		\State $\textbf{v}_l,\textbf{v}_r$$\leftarrow$ Interface velocities from cell $\textbf{i}$
		\State $dt_z\leftarrow\textsc{Diffusiontime}(d,\textbf{v}_t,\textbf{v}_r,L_c,D_m)$
		\If {$e_i\neq 0$} \Comment To avoid stream line starting point
		\State $i_{|e_i|}\leftarrow i_{|e_i|} + \frac{1}{2}(\frac{e_i}{|e_i|}+\frac{Z_{|e_i|}}{|Z_{|e_i|}|}) $
		\State $e_i \leftarrow -|e_i|\frac{Z_{|e_i|}}{|Z_{|e_i|}|}$ 
		\EndIf
		\Else \Comment Pure advective step
		\If {$e_i\neq 0$} \Comment To avoid stream line starting point
		\State $i_{|e_i|}\leftarrow i_{|e_i|} + \frac{e_i}{|e_i|} $ 
		\State $e_i \leftarrow -e_i$ 
		\EndIf
		\EndIf
		\State \textsc{Checkboundaries}
		\State \textsc{Checkgeometry}
		\State $(\textbf{v}_l,\textbf{v}_r)$$\leftarrow$ Interface velocities from cell $\textbf{i}$
		\State $(\textbf{x}_i,\textbf{x}_e,e_i)\leftarrow\textsc{Travelcell}(d, \textbf{v}_l, \textbf{v}_r, D_m, \textbf{Z}, dt_z, dt_e,L_c, u, a, dt, k)$ 
		\Until{$m\geq c_{max}$ or $t \geq t_{max}$}\Comment Move particle
		\EndProcedure
	\end{algorithmic}
\end{algorithm}

\begin{algorithm}[H]
	\caption{Diffusion step time}
	\label{alg:part_track}
	\begin{algorithmic}[1]
		\Procedure {Diffusiontime}{$d, v_l, v_r, D_m, \textbf{Z}, dt_z, dt_e,L_c, u, a, dt, k$}
		\EndProcedure
	\end{algorithmic}
\end{algorithm}

\begin{algorithm}[H]
	\caption{Dimensional advective/diffusive particle motion in a cell}
	\label{alg:travelcell}
	\begin{algorithmic}[1]
		\Procedure {Travelcell}{$d, \textbf{v}_l, \textbf{v}_r, D_{mi}, Z_i, dt_z, dt_{ze}, \textbf{dx}, \textbf{x}_i, \textbf{x}_e, \textbf{v}_e, \textbf{a}_e, dt, e_i, n_s$}
		\State $\textbf{A} \leftarrow \left(\frac{v_{r1}-v_{r1}}{dx_1},...,\frac{v_{rd}-v_{rd}}{dx_d}\right)^T$ 
		\State $\textbf{v}_1 \leftarrow \textbf{v}_l + \textbf{A}*\textbf{x}_i$ \Comment Current particle speed
		\For{$k=1,d$}
		\State $x_1\leftarrow (\textbf{x}_i)_k/dx_k,\quad v_1\leftarrow dx_k(\textbf{v}_1)_k/D_m$ \Comment Non dimensional
		\State $t_e \leftarrow \textsc{Traveltime}(x_1, v_1, Pe, t_e, x_e, n_s)$
		\EndFor
		\EndProcedure
	\end{algorithmic}
\end{algorithm}

\begin{algorithm}[H]
	\caption{Advective/diffusive time for crossing a cell}
	\label{alg:part_track}
	\begin{algorithmic}[1]
		\Procedure {Traveltime}{$x_1, v_1, Pe, Z, t_e, x_e$}
		\State $n_s\leftarrow 0$ \Comment Init. step counter
		\State $x_s=x_1 - v_1/\textrm{Pe}$ \Comment Stationary point with $u=0$
		\EndProcedure
	\end{algorithmic}
\end{algorithm}
%Anhang




%ABBILDUNGSVERZEICHNIS
% ***************************************************************************************************
%\clearpage
%\phantomsection
%\addcontentsline{toc}{chapter}{Abbildungsverzeichnis}
%\listoffigures

%TABELLENVERZEICHNIS
% ***************************************************************************************************
%\clearpage
%\phantomsection
%\addcontentsline{toc}{chapter}{Tabellenverzeichnis}
%\listoftables


\end{document}
