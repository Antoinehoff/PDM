\chapter{Introduction}
%%Contextual introduction
\begin{figure}
	\centering
	\includegraphics[scale=0.6]{Figures/bentheimer300_cuts.png}
	\caption{(left) Side cut of x-ray scanned sample of Bentheimer sandstone with $300^3$ voxel resolution and a side length of $L_0\approx0.3$mm. (right) Perpendicular averaged porosity for the same sample. Black regions show entirely solid column along the perpendicular axis.}
	\label{fig:bentheimer300cuts}
\end{figure}
Diffusion in porous media is a common but complex natural phenomenon occurring in a wide domain of application, from carbon storage in geologic layers to advanced oil extraction and medicine metabolization.
Despite of widely used macroscopic descriptions of transport such as the Fick's law of diffusion (\citet{Fick1855}) and Darcy law (\citet{Darcy1856}), complex porous media can present dispersion behaviors that require a microscopic approach in order to predict correctly the transport dynamic.
For example, the Bentheimer sandstone presents, at microscopic scale, very tortuous cuts with a small solid void ratio called porosity (c.f. Figure \ref{fig:bentheimer300cuts} left).
The complexity of this sample is also put in evidence on Figure \ref{fig:bentheimer300cuts} (right) by plotting the percentage of void cells over a perpendicular axis which illustrates the existence of completely blocked areas in the sample.
Flows among such media are driven by the pore-scale geometry and their description using simple macroscopic laws constitute an active research topic nowadays.\\
As an illustration, \citet{Dentz2017} demonstrated in their work, how the macroscopic advection-diffusion transport law
\begin{equation}
	\pdv{\bar{c}(x,t)}{t}+U\pdv{\bar{c}(x,t)}{x}-\mathcal{D}\pdv[2]{\bar{c}(x,t)}{x}=0
\label{eq:adv_diff}
\end{equation}
failed to describe the diffusion of a macroscale concentration $\bar{c}(x,t)$ for a mean flow $U$ and a macroscopic diffusion coefficient $\mathcal{D}$ by comparison to direct numerical simulation (DNS) data (c.f. \citet[Fig. 3]{Dentz2017}).\\


\section{General numerical approach of the problem}
In this article as in many other recent works, numerical approaches are adopted in order to bridge the gap between microscopic behavior of the flow and macroscopic diffusion as \citet{Meyer2016}, \citet{Dentz2017} and \citet{Puyguiraud2019}. The similarities in the approach chosen by these works have been combined to present the following common methodology.
\subsubsection{Sample scanning and physical features}
First of all, a digital porous domain is obtained either synthetically (\cite{Dentz2017}) or by high resolution tomography of physical sandstone samples (\cite{Meyer2016} and \cite{Puyguiraud2019}).
Hence, a scalar field  $\rho_d(\textbf{x})$ on a domain, $\Omega_d\ni\textbf{x}$, can be defined as 
\[
\rho_d(\textbf{x})=\left\{
\begin{array}{ll}
1 &  \textrm{if voxel at $\textbf{x}$ is solid} \\
0 & \textrm{otherwise}
\end{array} 
\right.
\]
In the wide variety of natural porous media, samples are usually differentiated through the the following features : the average pore length $L$, the porosity $\phi$ which expresses the ratio between solid and void volume, i.e.
\[
\phi = \frac{\int_{\Omega_d}\rho(\textbf{x})d\textbf{x}}{\int_{\Omega_d}d\textbf{x}}
\] 
and the tortuosity $\kappa$ which indicates how much paths in the medium are twisted by comparing at a time $t>0$, the total length of a path $l(t)$ with a linear travel distance $l_{\hat{\textbf{e}}_i}(t)$ in a arbitrary direction represented by a normal directional vector $\hat{\textbf{e}}_i$, i.e. 
\[
\kappa = \frac{l(t)}{l_{\hat{\textbf{e}}_i}(t)}.
\] 
As in most cases, this work will consider a cubic domain of side length $L_0$, $\Omega_d=[0,L_0]^3$, and use the Cartesian spatial coordinates $\textbf{x}=(x_1,x_2,x_3)^T=x_1\hat{\textbf{e}}_1+x_2\hat{\textbf{e}}_2+x_3\hat{\textbf{e}}_3$ in terms of the corresponding normalized directional vectors.\\
It is worth noting that the domain must balance two contradicting requirements in order to be viable for numerical experiments. 
On one hand, its size has to be large enough to be a representative elementary volume (REV) of the sample, i.e., the frequency of every possible configuration of the porous network should be measurable.
This property can be verified by increasing the size of an REV considered sample and verifying that the average properties do not change.
On the other hand, the domain should remain as small as possible to spare computational resources in the upcoming numerical treatments.\\

\subsubsection{Solving numerically the Stokes flow}
Once a REV of the porous sample has been digitalized, a numerical solver is used to obtain the stationary Eulerian velocity field $\textbf{u}_e$ from the incompressible Stokes equation,
\begin{equation}\label{eq:stokes}
\begin{array}{c}
	\hat{\mu}\nabla^2\textbf{u}_e(\textbf{x}) = \nabla p(\textbf{x}) \quad\textrm{and}\quad \nabla\cdot\textbf{u}_e(\textbf{x})=0\quad \forall \textbf{x}\in\Omega_d\\
	\textrm{s.t.}\quad p(L_0,x_2,x_3) - p(0,x2,x3) = \Delta\textrm{P} \quad \forall (x_2,x_3)\in[0,L_0]^2.
\end{array}
\end{equation}
where $\nabla = (\pdv{x_1},\pdv{x_2},\pdv{x_3})^T$ denotes the Cartesian gradient, $\hat{\mu}$\footnote{The hat has been added to the viscosity symbol in order to avoid confusion with the mean symbol, $\mu$, used further.} the fluid viscosity and $p(\textbf{x})$ the pressure field.
The pressure gap $\Delta\textrm{P}$ applied on two opposite faces of the cubic porous domain is meant to induce a mean flow velocity in the $\hat{\textbf{e}}_1$ direction\footnote{As a convention for this work, the $\hat{\textbf{e}}_1$ direction will always be the one aligned with the mean flow velocity and is called the longitudinal direction. By construction, the other directional vectors are regrouped under the transversal direction.}, 
\[
U=-\frac{L \phi \Delta\textrm{P}}{\hat\mu}.
\]
In order to account for the solid inside the porous domain, additional non slip boundary conditions have to be added in the solid domain, i.e.,
\begin{equation}\label{eq:nonslip_sideBC}
\textbf{u}_e(\textbf{x})=0 \quad\forall \textbf{x} \in \{\textbf{y}\in\Omega_d|\rho_d(\textbf{y})=1\}.
\end{equation}
One often refers to the velocity $U$ with the help of the Péclet number that expresses the ratio between advection due to the flow and molecular diffusion, i.e.
\begin{equation}\label{eq:peclet}
\textrm{Pe}=\frac{U\,L}{D_m}
\end{equation}
with $D_m$ the molecular diffusion coefficient. 
In this work, the Péclet number will be mostly used to express the value of the molecular diffusion coefficient $D_m$ since $U$ and $L$ are kept constant in the simulations.
Characteristic times can be defined as the macroscopic advection time $\tau_a=U/L$ (often used as reference time $T\equiv\tau_a$) and the microscopic diffusive time $\tau_D=L_0^2/D$.
These quantities are then used in the definition of the dimensionless variables
\begin{equation}
\bar{x}=x/L,\quad \bar{u}=u/U,\quad \bar{t}=t/T,\quad \bar{D}_m=1/\Pe.
\end{equation}
The numerical solution of (\ref{eq:stokes}) allows then to obtain a approximated map of the Eulerian velocity field $\textbf{u}$ in the digital porous domain, which enables to define the tortuosity of the studied domain as
\begin{equation}\label{eq:tortuosity}
\kappa=\frac{\int_0^t|\textbf{u}_e(t)|dt}{\int_0^t \textbf{u}_e(t)\cdot\hat{\textbf{x}}_idt}.
\end{equation}
By definition, the REV property ensures that the directional vector in \eqref{eq:tortuosity} can be arbitrary chosen without changing the value of the tortuosity.\\

\subsubsection{Direct numerical simulation of flowing particles}
Once an microscopic numerical solution of the flow in the medium is obtained, the displacement of $N_p$ particles injected at the inflow plane is obtained from another DNS using particle tracking methods as \citet{Pollock1988} to solve
\begin{equation}\label{eq:traject}
\dv{\textbf{x}_i(t)}{t}=\textbf{u}_e[\textbf{x}_i(t)]\quad\forall i\in\{1,2,...,N_p\}
\end{equation}
where $\textbf{x}_i(t)$ stands for the position of particle $i$ at time $t$.\\
Thus, a precise microscopic description of the particle dispersion among the sample is provided and macroscopic features as the macroscopic diffusion coefficient $\mathcal{D}$ can be estimated for various flow conditions.\\
Unfortunately, solving equation (\ref{eq:stokes}) and simulating a large number of particles flowing through the sample is costly in term of computational resources. Hence, it is hardly thinkable to use DNS methods for obtaining efficient descriptions of diffusion in a wide variety of porous media.\\
This issue motivated a probabilistic approach of the microscopic particles dynamic with the help of stochastic processes like the historical success of the Brownian motion modeling by \citet{Einstein1906}. 
In his work, \citeauthor{Einstein1906} predicted accurately the displacement probability of a pollen grain over time without considering a complex deterministic model that would have included an exhaustive description of particle to particle interactions.\\
In that sense, being able to build stochastic models that reproduce precisely the DNS output is of great importance to efficiently predict complex dispersion phenomena.

%Mathematical tools for stochastic processes
\section{Mathematical tools for stochastic modeling}
In a general way, a stochastic process can be seen as the existence of a certain time dependent random variable $\textbf{X}(t)$ in a system (c.f. \citet[Sec. 3.1]{Gardiner1996}). The probability for $\textbf{X}(t)$ to visit successively the $N_x$ states $\{\textbf{x}_i,t_i\}_{i=1}^{N_x}$ knowing the $N_y$ previously visited states $\{\textbf{y}_j,\tau_j\}_{j=1}^{N_y}$  can be expressed as
\begin{equation}\label{eq:cond_prob}
p(\{\textbf{x}_i,t_i\}_{i=1}^{N_x}|\{\textbf{y}_j,\tau_j\}_{j=1}^{N_y})\quad\textrm{where}\quad \tau_1<\tau_2<...<\tau_{N_y}<t_1<t_2<...<t_{N_x}.
\end{equation}
where $p(a|b)$ stands for the probability of $a$ if $b$ happened.
Most stochastic processes are called separable, i.e., the probability (\ref{eq:cond_prob}) defines entirely the evolution of $\textbf{X}(t)$.

\subsubsection{Markov processes}
Markov processes are separable stochastic processes, widely used in physics, with conditional probability \eqref{eq:cond_prob} that depends only on the current state $(\textbf{x},t):=(\textbf{y}_{N_y},\tau_{N_y})$ and not on the previously visited ones, i.e. 
\[
p(\{\textbf{x}_i,t_i\}_{i=1}^{N_x}|\{\textbf{y}_j,\tau_j\}_{j=1}^{N_y})=p(\{\textbf{x}_i,t_i\}_{i=1}^N|\textbf{y}_{N_y},\tau_{N_y}).
\]
A system modeled by these so-called Markov processes can be seen as if the previous visited states are not kept in memory, which is a reasonable statement for numerous physical systems, e.g., as in flowing particles or molecular dynamics.
\subsubsection{Stochastic differential equations and probability distribution functions}
The evolution of a time dependent, one dimensional, random variable $X(t)$ that follows a Markov process can be expressed as a general stochastic differential equation (SDE) of the form
\begin{equation}
	dx = a(x,t)dt + b(x,t)\xi(t)dt
	\label{eq:langevin}
\end{equation}
called the Langevin equation (\citet[Eq. 4.1.1]{Gardiner1996}). 
On the right hand side, $a(x,t)$ and $b(x,t)$ are drift and diffusion functions, respectively and $\xi(t)$ is an uncorrelated Gaussian variable, i.e.,
\[
p(\xi(t)=x)=\frac{1}{\sqrt{2\pi}\sigma}\exp(-\frac{(x-\mu)^2}{2\sigma^2})
\]
with mean $\mu=0$ and variance $\sigma^2=1$.\footnote{In this work, the symbol $\mu$ and $\sigma^2$ will always refer to the mean and variance, respectively, while the standard deviation will refer to the quantity $\sigma$}\\
The simplest case of a Markov process is the pure random walk (RW)
\begin{equation}\label{eq:RW}
	dx = dW
\end{equation}
where $dW\equiv\xi(t)dt$.
If we add a drift $\nu=\mathrm{const}$ to (\ref{eq:RW}), we obtain the Wiener process
\begin{equation}\label{eq:wiener}
dx = \nu dt + dW
\end{equation}
that can be seen as an oriented random walk where $\nu$ sets the direction and the shifting speed of the average.\\
The use of an $x$ dependent drift leads to a commonly used Markov process, the Ornstein-Uhlenbeck (OU) process (\citet[Sec. 3.8.4]{Gardiner1996})
\begin{equation}
dx = -kxdt + \sqrt{D}dW,
\label{eq:OU}
\end{equation}
which can be identified as a Langevin equation (\ref{eq:langevin}) with $a(x,t)=a(x)=kx$, $b(x,t)=\sqrt{D}=\mathrm{const}$. 
One can prospect that the sign of $k$ can change the behavior of $X(t)$ drastically from a convergence into a pure random walk ($k>0$) to divergence ($k<0$).\\
With the help of Ito's formula (see \citet[Sec. 4.3.4]{Gardiner1996}), one can obtain the differential equation that governs the evolution of the probability density function (PDF) $p(x,t)$ for any process of the form (\ref{eq:langevin}),
\begin{equation}
\partial_t p(x,t) = \partial_x[a(x,t)p(x,t)] + \frac{1}{2}\partial^2_x [b(x,t)p(x,t)]
\label{eq:fokkerplanck} 
\end{equation}
called the Fokker-Planck equation. Equation (\ref{eq:fokkerplanck}) can be analytically solved for each processes seen before, hence leading to analytical expressions for the mean and the variance of the random variable. 
In case of an OU process (\ref{eq:OU}), an explicit calculation (\cite[Sec. 3.8.4]{Gardiner1996}) leads to
\[
\langle X(t)\rangle=x_0 \exp(-kt) \quad\textrm{and}\quad \langle[X(t)-\langle X(t)\rangle]^2\rangle=\frac{D}{2k}[1-\exp(-2kt)]
\]
for the mean value and the standard deviation, respectively. One can see that, as $t\rightarrow\infty$, the OU process tends to a steady Gaussian PDF of zero mean and $D/2k$ variance which gives insights of its utility for velocity modeling in diffusion. \\
In fact, equation (\ref{eq:fokkerplanck}) can be generalized for multi-dimensional random variables $\textbf{Z}(t)\equiv(z_1(t),z_2(t),...,z_d(t))^T$ as (\citet[Sec. 3.5.2]{Gardiner1996}) 
\begin{equation}
\frac{\partial p(\textbf{z},t|\textbf{y},t')}{\partial t} =
-\sum_{i=1}^d\frac{\partial}{\partial z_i}[A_i(\textbf{z},t)p(\textbf{z},t|\textbf{y},t')]
+ \frac{1}{2}\sum_{i,j=1}^d\frac{\partial^2 }{\partial z_i \partial z_j}[B_{ij}(\textbf{z},t)p(\textbf{z},t|\textbf{y},t')]
\label{eq:mdfp}
\end{equation}
where $t$ and $t'$ denote two different times, $\textbf{y}$ and $\textbf{z}$ two different values of the random vector $\textbf{Z}$.\\ 
We can see now that drift and diffusion are expressed in a general manner by the vector $\textbf{A}(\textbf{z},t)$ and the matrix $\underline{\textbf{B}}(\textbf{z},t)$, respectively.\\
Thus, one of the major goals of actual research in porous media can be resumed as finding the functions $\textbf{A}(\textbf{z},t)$ and $\underline{\textbf{B}}(\textbf{z},t)$ using geometry based arguments (e.g. pore length $L$ and porosity $\phi$) and macroscopic features of the flow (e.g mean velocity $U$ and pressure gap $\Delta\textrm{P}$) in order to build a stochastic model that could reproduce non Fickian diffusion dynamic and transition to Fickian regime.\\

\section{Estimation of PDF by sampling in particle trajectory}
In the case of flowing particles, modeling the velocity of a particle $i$ as a Markovian variable $\textbf{U}_i$ and knowing a state $(\textbf{x}_0,\textbf{u}_0,t_0)$ would entirely define its position time serie 
\[
\textbf{X}_i(t)=\int_{t_0}^{t}\textbf{U}_i(t')dt',
\]
which is an equivalent expression of (\ref{eq:traject}).
Hence, one needs to estimate Lagrangian velocity PDF $p(\textbf{u},t)$ from empirical data to set the process parameters, e.g., $D$ and $k$ of \eqref{eq:OU}. 
\\The PDF of a random variable $Q(t)$ can be exactly defined by sampling over an infinitely large amount of $N_p$ particle
\[
p(Q(t)=q)\equiv p(q,t)=\lim\limits_{N_p\rightarrow\infty}p_{N_p}(q,t) \quad\textrm{where}\quad p_{N_p}(q,t)=\frac{1}{N_p}\sum_{i=1}^{N_p}\delta(q-q_i(t))
\]
In this relation, the approximated PDF $p_{N_p}(q,t)$ can now be obtained easily thanks to the DNS time series $\{q(\textbf{a}_i,t)\}_{i=1}^{N_p}$ mesured from the $N_p$ particles and differentiated here by their respective initial position $\{\textbf{a}_i\}_{i=1}^{N_p}$.\\

However, in order to provide an accurate estimation of $p(q,t)$ with a finite number of particles, the DNS data 
must be sampled during a sufficiently long simulation time $T_{sim}$ and for a sufficient number of particles. 
Mathematically, these sufficiency conditions are formally expressed as ergodicity conditions on the particle time series, i.e.,
\begin{equation}
\begin{array}{c}
	\langle q(\textbf{a}_i,t)\rangle_t = \langle q(\textbf{a},t_n)\rangle_{\textbf{a}}\\
	\Leftrightarrow\\
	\int_{0}^{\infty}q(\textbf{a}_i,t) p(q(\textbf{a}_i,t),t)dt=\int_{-\infty}^{\infty}q(\textbf{a},t_n)p(q(\textbf{a},t_n),t_n)d\textbf{a}
\end{array}
\label{eq:ergodicity}
\end{equation}
One can understand ergodicity as the equivalence between averaging over an infinite number of particles at a given time $t_n$ and averaging over an infinite time for an arbitrary particle $i$. 
More practically, ergodicity condition with a finite number of particle can still be tested for any particle $i$ at any time $t_n$ with the error
\begin{equation}\label{eq:ergo_err}
	e(\textbf{a}_i,t_n)=|\frac{1}{N_T}\sum_{k=1}^{N_T}q(\textbf{a}_i,t_k)-\frac{1}{N_p}\sum_{j=1}^{N_p}q(\textbf{a}_j,t_n)|.
\end{equation}
Alternatively, it can be verified by testing that the approximated distribution does not change when increasing the size of the serie.\\
Thus, ergodicity is the most important condition to respect in order to estimate accurately PDF with DNS data. 
The Markov model parameters can then be set in order to fit the DNS data PDF with random variables PDF under ergodic condition.\\
Thus, the procedure described above forms the basis of the studied papers in the following short literature review.
\section{Literature review of recent results}
\subsubsection{Puyguiraud et al. 2019}
In their latest work, \citet{Puyguiraud2019} presented an analysis of Lagrangian pore-scale velocity series in a three-dimensional sample of a Berea sandstone, well known for retaining oil and natural gas.
Their numerical porous domain was a cube of $L_0^3=0.95$mm$^3$ volume, $L=0.15$mm average pore length and $\kappa=1.75$ tortuosity, obtained with a scanning resolution of $300^3$ voxels\footnote{The porosity of their sample has not been precised. Experimentals results on Berea sandstone showed a porosity that can vary between $10$\% to $25$\%.}. 
With the help of the \textsc{SIMPLE} algorithm of \textsc{OpenFoam} and a mesh of $900^3$ cubes that fitted perfectly the scan voxels, the single-phase pore-scale flow has been obtained by solving the Stokes equation (\ref{eq:stokes}) with non slip-boundary conditions on the solid surface and on the longitudinal sides of the cube, i.e., 
\begin{equation}\label{eq:nonslip_sideBC}
\textbf{u}_e(\textbf{x})=0 \quad\forall \textbf{x}\in \{\textbf{y}\in\Omega_d|y_2,y_3\notin]0,L_0[\}.
\end{equation}
resulting in a mean flow velocity $U=8.05\cdot 10^{-4}$m/s.\\
Then, \citeauthor{Puyguiraud2019} investigated the evolution of $N_p=10^6$ particles injected as a uniform $\rho_u$ or flux-weighted $\rho_{fw}$ distribution on an initial particle domain $\Omega_0$, i.e.,
\begin{equation}\label{eq:unif_distr}
\rho_u(\textbf{a})=\frac{\mathcal{I}(\textbf{a}\in\Omega_0)}{\int_{\Omega_0} d\textbf{a}}
\end{equation}
and
\begin{equation}\label{eq:flux_weighted}
\rho_{fw}(\textbf{a})=\frac{\norm{\textbf{u}(\textbf{a})}_2\mathcal{I}(\textbf{a}\in\Omega_0)}{\int_{\Omega_0} \norm{\textbf{u}(\textbf{a})}_2 d\textbf{a}}
\end{equation}
respectively, where $\mathcal{I}(a)=1$ if $a$ is true and $0$ otherwise and $\textbf{u}$ the Lagrangian velocity. 
The trajectories equation (\ref{eq:traject}) has been solved for each particle using a quadratic interpolation of the face velocities and the particles that reached the outlet plane with a speed $u_{L_0}\equiv\norm{\textbf{u}[(L_0,x_2,x_3)^T]}$ were reinjected randomly among the set of position on the inlet plane
\[
\Omega_{in}=\{\textbf{x}\in\Omega_d|\,x_1=0,\,\norm{\textbf{u}(\textbf{x})}_2\in[u_{L_0}\pm\Delta u]\}\quad\rm{with}\quad \Delta u=u_{L_0}/200.
\]
During the trajectories computation, two different Lagrangian velocity statistics have been sampled from each particle: one sampled with constant time intervals, i.e. isochronous t-Lagrangian velocity statistics
\[
\{u(\textbf{a},t_{n})\}_{n=0}^{N_T}\quad\textrm{with $\{t_n\}_{n=0}^{N_T}$ s.t.}\quad \Delta t_n\equiv t_{n+1}-t_n = \Delta t=\mathrm{const},
\] 
the other sampled with constant space intervals, i.e. equidistant s-Lagrangian velocity statistics
\[\{u(\textbf{a},s_{n})\}_{n=0}^{N_S}\quad\textrm{with $\{t_n\}_{n=0}^{N_S}$ s.t.}\quad \Delta s \equiv\norm{\textbf{u}(t_n)}_2\Delta t_n = \mathrm{const}.
\]
After a graphical comparison between the s- and t-Lagrangian velocity series (see \citet[Fig. 2]{Puyguiraud2019}), intermittency in the isochronous data has been taken as a motivation to model uniquely the equidistant velocity series.\\
Then, they considered their s-Lagrangian velocity serie as a stationary ergodic Markov process. 
To demonstrate the correctness of this hypothesis, they modeled the dynamic of the s-Lagrangian PDF $p(u,s)$ with a process that followed empirical, discrete transition probabilities $T_{nm}\equiv r(u_m,s_m-s_n|u_n)$ obtained from the DNS data as
\[
T_{nm}=\frac{1}{N_p}\sum_{i=1}^{N_p}\frac{1}{N_S}\sum_{k=0}^{N_S}\mathcal{I}[u_n \leq u_i(s_n) < u_n + \Delta u_n]\cdot\mathcal{I}[u_m \leq u_i(s_n+\Delta s) < u_m + \Delta u_m]
\]
where $u_i(s_n)\equiv u(\textbf{a}_i,s_n)$ and $\{\Delta u_l\}_{l=1}^{N_u}$ represent a division of the velocity range in $N_u$ bins s.t. $\min[u(\textbf{a},s)]=u_0$ and $\max[u(\textbf{a},s)]=u_{0}+\sum_{l=1}^{N_u}\Delta u_l$.\\
The empirical transition probability matrix $T_{nm}$ showed notable variations for $\Delta s\in[\frac{1}{150}L,\frac{8}{3}L]$ as displayed on \citet[Fig. 7]{Puyguiraud2019}. 
This has been interpreted as a consequence of the non exponential character of the Lagrangian velocity correlation function for distances smaller than a pore length $L$, which motivated \citeauthor{Puyguiraud2019} to set the equidistant sampling rate at $\Delta s = 8L/3$.\\
The dynamic of the s-Lagrangian velocity PDF resulting from $10^8$ empirical Markov processes reproduced accurately the evolution of the same PDF for $10^7$ particles tracked along $3.8L_0$, i.e $3.8L_0/\kappa U\approx 10^{-1}T$, initialized with a flux weighted distribution.\\
Thereafter, they relaxed their fully empirical Markov model by using a Bernoulli process with transition probability
\begin{equation}\label{eq:bernoulli}
r(u,\Delta s|u')=P(u) + [1 - P(u)] e^{-\frac{\Delta s}{l_c}},
\end{equation}
where $l_c$ is a cutting length parameter and $P(u)$ is the steady state probability distribution. 
This model did not reproduced the evolution of $p(u,s)$ over time as the empirical one but \citeauthor{Puyguiraud2019} used it to set the cutting length $l_c=2.5L$ by adjusting the convergence of the mean velocity with DNS data for large $s$. \\
The identified cutting length has then been used to construct an OU process $w(s)$
\[
dw = -l_c^{-1}wds + \sqrt{2l_c^{-1}}\xi(s)ds
\]
solved numerically for an ensemble of $10^7$ particles using the explicit Euler scheme 
\[
w_{n+1}=w_n-l_C^{-1}w_n\Delta s + \sqrt{2l_c^{-1}\Delta s}\xi_n
\]
where $f_n$ denotes $f(n\Delta s)$.
The value $w(s)$ has been linked to the velocity $u(s)$ thanks a Smirnov transform that enforced the equality between the cumulative distributions of $w$ and $v$, $\Pi(u)$ and $\Phi(w)$ respectively. 
Hence, the next step velocity of particle $i$, $u_{n+1}^i\equiv u(\textbf{a}_i,(n+1)\Delta s)$, can be obtained from $w_{n+1}$ as
\[
\begin{array}{rrcl}
\Pi(u_{n+1}) = \Phi(w_{n+1}) \Leftrightarrow& \int_0^{u_{n+1}} P(u')du' &=& \int_{-\infty}^{w_{n+1}}\Phi(w')dw'\\
\Rightarrow& \frac{1}{n+1}\sum_{m=0}^{n+1}u^i_m&=&\frac{1}{n+1}\sum_{m=0}^{n+1}w^i_m\\
\Rightarrow & u^i_{n+1} &=& \sum_{m=0}^{n+1}w^i_m - \sum_{m=0}^{n}u^i_m.
\end{array}
\]
With this coupled OU model, \citeauthor{Puyguiraud2019} succeeded to model the dynamic of the ensemble s-Lagrangian velocity PDF for uniform and flux weighted initial particle distributions.
To conclude their work, they affirmed that this latter model has been parameterized using hydraulic and geometric features of the pore media. 
However, it would be interesting to know more about the sensitiveness of the OU process with respect to the cutting length since a geometric argumentation of $l_c=2.5L$ has not been provided.
The versatility of their best stochastic model could have been tested for other porous media as well.

\subsubsection{Dentz et al. 2017}
On their side, \citet{Dentz2017} studied a synthetic porous media built by simulating sedimentation of irregularly shaped grains generated with an empirically fitted Weibull distribution
\begin{equation}\label{eq:weibull}
	p_W(x)=bkx^{k-1}\exp(bx^k)
\end{equation}
with parameters $b$ and $k$.\\
They obtained a cubic sample of length $L=2$mm, porosity $\phi=35\%$, pore length $L_0=0.28$mm, tortuosity $\kappa=1.6$ and applied a pressure gradient at the boundaries that imposed a mean flow velocity $U=5.73\times 10^{-6}$m/s. By changing the value of the microscopic diffusion coefficient $D$, they analyzed their sample on a range of Péclet number $Pe\in\{30, 3\times 10^2, 5\times 10^2, 10^3\}$.\\
As said in the beginning of this introduction, \citeauthor{Dentz2017} demonstrated the unavailability of the macroscopic model \eqref{eq:adv_diff} in complex porous medium by comparison of the analytical solution with a pore-scale Eulerian explicit DNS solving the Stokes equations \eqref{eq:stokes} coupled with the local transport PDE
\begin{equation}\label{eq:micro_adv_diff}
\pdv{t}c(\textbf{x},t)+\nabla\cdot[c(\textbf{x},t)\textbf{u}_e(\textbf{x},t)]-D \nabla^2c(\textbf{x},t)=0.
\end{equation}
As evaluation criterion, they defined the breakthrough curves
\begin{equation}
f(x,t)=1-\frac{\int_0^L\int_0^L\,c|_{x_1=x}\textbf{u}_e|_{x_1=x}\cdot \hat{\textbf{e}}_1 dx_2\,dx_3}{\int_0^L\int_0^L\,\textbf{u}_e|_{x_1=x}\cdot \hat{\textbf{e}}_1dx_2\,dx_3},
\label{eq:DentzBTC}
\end{equation}
which expresses the particle-free fraction of the outlet plane when $x=L$.\\
They showed that the macroscopic model \eqref{eq:adv_diff} was not able to capture the tailing behavior of the pore-scale description either at intermediate or long time ($t/T\approx 10^0$ and $\approx 10^2$, respectively).\\
Thereafter, \citeauthor{Dentz2017} provided a physical model of the Eulerian velocity magnitude PDF $p_e(u_e\equiv\norm{\textbf{u}_e(\textbf{x})}_2)$  obtained from the DNS data. 
They modeled the PDF as a sum of pore contributions, assumed as local parabolic Poiseuille flows 
\begin{equation}\label{eq:dentz_poiseuille}
u_{pf}(r)=\left(\frac{d_p^2-(2r)^2}{d_0^2}\right)u_0
\end{equation}
where $r$ is the radial position with its origin at the center of the pseudo cylindrical pore, $d_p$ is the pore diameter, $u_0$ and $d_0$ are a characteristic velocity and diameter, respectively.
They assumed that the pore diameter distribution $p_d(d)$ was similar to their grain size distribution, i.e., a Weibull distribution \eqref{eq:weibull} with parameters $k=\alpha$ and $b=(d_0)^{-\alpha}$
\[
p_d(d)=\frac{\alpha}{d_0}\left(\frac{d}{d_0}\right)^{\alpha-1}\exp[-(d/d_0)^\alpha].
\]
Hence, they achieved to obtain an analytical expression for the Eulerian velocity PDF :
\begin{equation}\label{eq:dentz_pev}
p_e(u)=\frac{\exp[-(u/u_0)^{\alpha/2}]}{u_0\Gamma(1+\alpha/2)}
\end{equation}
with $\alpha\approx2.5$ and $v_0\approx2.1U$ estimated by fitting with the DNS Eulerian velocity PDF.
\citeauthor{Dentz2017} also estimated the mean pore velocity distribution $p_m(v)$ in a similar way and approximate it with a gamma distribution
\begin{equation}\label{eq:dentz_pmv}
p_m(u)=\frac{\beta}{U}\left(\frac{\beta u}{U}\right)^{\alpha/2}\exp\left(-\frac{\beta u}{U}\right),
\end{equation}
where $\beta=(\alpha/2+1)$.\\
It is worth noting that their model \eqref{eq:dentz_pmv} showed good agreement with empirical data by using a unique parameter. Of course, it would be interesting to see if it can be extrapolated to other porous geometries in order to test truly its efficiency.\\
After this analytical modeling, \citeauthor{Dentz2017} continued their investigation by using a particle based simulation ($N_p=10^9$) of the transport among the synthetic porous domain for simulation time up to $t/T=10^2$ and tried to model the resulting breakthrough curve with incrementally complex time domain random walks (TDRW) similarly to \citeauthor{Puyguiraud2019}.\\
Their first model considered purely advective processes, $\mathrm{Pe}=\infty$, with equidistant space transitions by defining the transition time as $\Delta t_n=l/u(t_n)$ with a transition length around the average pore size ($l\sim L$). 
They estimated analytically the asymptotic behavior of the breakthrough curve $f(x,t\gg T)\propto t^{-2}$ for the purely advective case by approximating the transition times PDF with a flux weighted Eulerian velocity PDF \eqref{eq:dentz_pev}.\\
%\[\psi_a(t)=\frac{l}{t^3U}p_e(l/t)\]
The predicted $t^{-2}$ dependence of the breakthrough curve for purely advective transport was confirmed by the particle simulation for any Péclet value Pe$\in\{30, 3\times 10^2, 5\times 10^2, 10^3\}$. However this TDRW model did not capture the breakthrough curve tailing at either intermediate nor long time.\\
In the second model, they included diffusion by considering the Lagrangian velocity PDF as a flux weighted pore mean velocity PDF \eqref{eq:dentz_pmv}. This hypothesis was motivated by the fact that particles should experience a pore-scale Poiseuille flow \eqref{eq:dentz_poiseuille} that is smoothed by diffusion.
The direction of spatial transition was also randomized to allow upstream transitions with probability 
\begin{equation}\label{eq:dentz_wu}
w_u(u)=\frac{D\tau_v}{l^2}\sim \Pe^{-1}.
\end{equation}
where $\tau_v$ is a characteristic diffusion time.\\
This advective-diffusive TDRW model showed a breakthrough curve in good agreement with DNS data for the studied Péclet range up to long time where it failed to capture the tailing behavior. \\
To catch the behavior of the DNS breakthrough curve at long time, \citeauthor{Dentz2017} changed their advective-diffusive TDRW by using an arbitrary small transition length $\Delta s\ll L$. 
They also replaced the pore-scale features used in the previously estimated PDF, i.e., the pore-scale velocity $v$ and diffusion coefficient $D$, by macroscopic ones, i.e., the mean Eulerian velocity $U$ and the hydrodynamic diffusion coefficient $\mathcal{D}$.\\
Moreover, they modeled the probability for a particle to be trapped in low velocity regions with a Poisson distribution. 
The parameters of the process were estimated from the DNS breakthrough curve at long time resulting in $\gamma\approx1.25\times 10^{-4}\tau_D$ and $\langle\tau_f\rangle\approx\frac{1}{2}\tau_D$ for the rate and the mean trapping time, respectively.\\
By including diffusion in the previous macroscopic trapping TDRWs, \citeauthor{Dentz2017} succeeded in reproducing the complete behavior of the DNS breakthrough curve with a stochastic model for $Pe\in\{30, 3\times 10^2, 5\times 10^2, 10^3\}$. This final model contained five main parameters,
\[
u_0=2.1U,\quad\alpha=2.5,\quad l(\mathrm{Pe})=f(\mathrm{Pe})L,\quad\gamma\tau_D=2.5\times 10^{-4}\quad\textrm{and}\quad\langle\tau_f\rangle/\tau_D=0.5,
\]
that can be counted as nine if one takes into account that the step length $l$ has been slightly varied for each Péclet value ($f(\mathrm{Pe}=[30, 3\times 10^2, 5\times 10^2, 10^3])=[0.7,0.77,0.77,1]$) without detailed explanations.\\
Like \citeauthor{Puyguiraud2019}, \citeauthor{Dentz2017} focused their work on the equidistant Lagrangian velocity magnitude and for a unique porous sample. However, the stochastic models have been tested for different flow features of the dispersion mechanism. 
One can see the breakthrough curve of \citeauthor{Dentz2017} as a finer tool than the Lagrangian velocity PDF evaluation used in \citeauthor{Puyguiraud2019} since it requires to accurately take into account the particles trapped in slow velocity regions.
It would be interesting to measure the breakthrough curve resulting from the OU process of \citet{Puyguiraud2019} for a better comparison  of their results.
It can be expected that the TDRW would show better performances than the OU process since the former required to set five parameters and the latter only one.

\subsubsection{Meyer et al. 2016}
It is interesting to relate both previously cited papers to the work of \citet{Meyer2016} who focused on isochronous time Lagrangian velocity PDF obtained from five different physical samples. \\
The main purpose of their study was to test the ability of a previously developed Markov process to universally describe transport in porous domain.
In order to do so, they took several physical porous samples with various geometric features, from the simple bead pack structure with $\kappa^{\rm{BP}} = 1.26$, $L^{\rm{BP}}_0=0.6$mm and $\phi^{\rm{BP}}=35.9\%$, to complex Estaillades structure characterized by $\kappa^{\rm{Est}} = 2.26$, $L_0^{\rm{Est}}=3.31$mm and $\phi^{\rm{Est}}=10.9\%$.\\
The flow inside each porous medium has been solved for a pressure gap $\Delta\textrm{P}=1$Pa. 
Inversely to \citet{Dentz2017}, molecular diffusion was not included in the DNS resulting in a fully advective regime ($\Pe=\infty$).\\
\citeauthor{Meyer2016} then modeled the Lagrangian velocity PDF extracted from isochronous time series as skew-normal PDF regarding the log velocity magnitude, i.e.
\begin{equation}\label{eq:meyer_skew}
p(v)=\frac{1}{\sqrt{2\pi}\sigma}\exp\left[-\frac{(v-\mu)^2}{2\sigma^2}\right](1-\erf\left[-\frac{\alpha(v-\mu)}{\sqrt{2}\sigma}\right])
\end{equation}
where $\alpha$ is the skewness parameter and $v=v(t)\equiv\ln[\norm{\textbf{u}(t)}_2/U]$ the log velocity magnitude\footnote{As a convention for this work, the letter $v$ will always refe to the Lagrangian velocity magnitude.}.
This PDF enabled to build a stochastic description of the Lagrangian velocity evolution using a Markov process following a Langevin equation \eqref{eq:langevin} with 
\begin{equation}\label{eq:meyer_model_v}
b^2(v,t)=d(v)=c\exp[bv] \quad\rm{and}\quad a(v,t)=a(v)=\frac{1}{2}d(v)\ln[p(v)d(v)]
\end{equation}
resulting in a five parameters model ($b,c,\alpha,\mu,\sigma$).\\
They also provided an analytical model for the directional angle $\theta=\cos^{-1}(\frac{\textbf{u}\cdot\textbf{U}}{\norm{\textbf{u}}_2\norm{\textbf{U}}_2})$,
\[p(|\theta|)=\frac{|\theta|}{\sigma_\theta^2}\exp\left[-\frac{|\theta|}{2\sigma_\theta^2}\right]\]
where the relation between $\sigma_\theta$ and the tortuosity, i.e. $\kappa=1/\langle cos(|\theta|)\rangle$, was confirmed by the DNS data. This PDF is then linked to an oscillatory stochastic process with the help of a Smirnov transform used in the same way as in \citet{Puyguiraud2019}. \\
Thanks to this analysis, \citeauthor{Meyer2016} showed that the traveled distance $L_\theta$ with constant $\theta$ sign could be efficiently represented by an exponential PDF for all media considered (c.f. \citet[Fig- 12]{Meyer2016}). They also modeled a second directional angle $\beta$ with an OU process, identified by analyzing the DNS data.
With the help of the five parameters Markov modeling of the velocity magnitude combined with the oscillatory and OU stochastic models for the directional angles, \citet{Meyer2016} were able to generate three-dimensional trajectories that could be confused with DNS ones. \\
More quantitatively, they tested the performance of their stochastic models by measuring the particle position PDF over time and its variance for $2\times 10^5$ particles in both longitudinal and transversal directions.
It appeared that the predicted dispersion was more accurate for medium with low tortuosity and high porosity samples. 
\cite{Meyer2016} explained it by the fact that samples of complex media need to be larger in order to be considered as an REV which would drastically increase the computational cost though.\\

It can be hard to compare the samples studied by \citeauthor{Meyer2016} with the Berea sandstone of \citet{Dentz2017} and the synthetic samples of \citet{Puyguiraud2019}. 
The Bentheimer structure of \citet{Meyer2016} is surely the most similar to both media in term of tortuosity with $\kappa_{\rm{Bent}}=1.65$. 
However, its porosity does not match ($\phi_{\rm{Bent}}=21.1\%$ against $\phi_{\rm{Berea}}=35\%$).
In any case, the comparison between various media done by \citet{Meyer2016} tended to show that structures with low porosity and high tortuosity were the most complex to model. 
It would be interesting to try to extract and test an approximated law that links the model parameters to the geometric features of the medium.\\

In a more general point of view, the three presented papers showed three different ways to test the results obtained from stochastic modeling. 
Since the Lagrangian velocity PDF rules the displacement of the particles at a microscopic scale, it seems clear that the model should reproduce the behavior of this distribution first.
Isochronous sampling displayed complex dynamics in both \citet{Meyer2016} and \citet{Puyguiraud2019} works, which may indicate that it contains, generally, more information about the structure of the flow than equidistant series.
In particular, slow velocity particles may be underestimated by equidistant sampling which can be a problematic bias when trying to model accurately the particle displacement in a physical point of view.\\
The comparison between particle breakthrough curves done by \citet{Dentz2017} over time is a very good way to obtain condensed results without losing much information. However, it becomes useless as soon as the considered plane is saturated.\\
The comparison criteria chosen by \citet{Meyer2016} are more exhaustive as it shows time evolution of particle position PDF but may be harder to read.\\

Inspired from the presented papers, this work aims to study the transition between fully advective to advection-diffusion regimes in porous media by analyzing isochronous time series obtained from direct numerical simulations. 
Finding the Péclet number as well as measuring the size of a possible transient area seem to be relevant in order to quantitatively measure the effect of molecular diffusion.\\
The results will then be explained with a physical reasoning on the microscopic structure of the flow which shall lead to some conclusion about the role of diffusion for transport in porous media.\\
In order to do so, the Bentheimer sample is chosen as the angular stone of this work as it showed an intermediate complexity in the work of \citet{Meyer2016}.
